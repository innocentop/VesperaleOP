\documentclass[11pt,twoside]{book}

%%Page Size (rev. 08/19/2016)
%\usepackage[inner=0.5in, outer=0.5in, top=0.5in, bottom=0.5in, papersize={6in,9in}, head=12pt, headheight=30pt, headsep=5pt]{geometry}
\usepackage[inner=0.5in, outer=0.5in, top=0.5in, bottom=0.5in, papersize={5.5in,8.5in}, head=12pt, headheight=30pt, headsep=5pt]{geometry}
%% width of textblock = 324 pt / 4.5in
%% A5 = 5.8 x 8.3 inches -- if papersize is A5, then margins should be [inner=0.75in, outer=0.55in, top=0.4in, bottom=0.4in]


%%Header (rev. 4/11/2011)
\usepackage{fancyhdr}
 \pagestyle{fancy}
\renewcommand{\chaptermark}[1]{\markboth{#1}{}}
\renewcommand{\sectionmark}[1]{\markright{#1}}
 \fancyhf{}
\fancyhead[LE,RO]{\thepage}
\fancyhead[CE]{\leftmark}
\fancyhead[CO]{\rightmark}
 \fancypagestyle{plain}{ %
\fancyhf{} % remove everything
\renewcommand{\headrulewidth}{0pt} % remove lines as well
\renewcommand{\footrulewidth}{0pt}}



\usepackage[autocompile,allowdeprecated=false]{gregoriotex}
\usepackage{gregoriosyms}
\gresetgregoriofont[op]{greciliae}




%%Titles (rev. 9/4/2011) -- TOCLESS --- lets you have sections that don't appear in the table of contents

\setcounter{secnumdepth}{-1}

\usepackage[compact,nobottomtitles*]{titlesec}
\titlespacing*{\chapter}{0pt}{-30pt}{0pt}
\titlespacing*{\section}{0pt}{*0}{*1}
\titlespacing*{\subsection}{0pt}{*0}{*1}
\titlespacing*{\subsubsection}{0pt}{*0}{*1}
\titleformat{\chapter} {\normalfont\LARGE\sc\center}{\thechapter}{0pt}{}
\titleformat{\section} {\normalfont\Large\sc\center}{\thesection}{1em}{}
\titleformat{\subsection} {\normalfont\large\sc\center}{\thesubsection}{1em}{}
\titleformat{\subsubsection}{\normalfont\normalsize\sc\center}{\thesubsubsection}{1em}{}

\newcommand{\nocontentsline}[3]{}
\newcommand{\tocless}[2]{\bgroup\let\addcontentsline=\nocontentsline#1{#2}\egroup} %% lets you have sections that don't appear in the table of contents


%%%

%%Index (rev. December 11, 2013)
\usepackage[noautomatic,nonewpage]{imakeidx}

\usepackage{multicol}

\makeindex[name=incipit,title=Index]
\indexsetup{level=\section,toclevel=section,noclearpage}

\usepackage[indentunit=8pt,rule=.5pt,columns=2]{idxlayout}


%%Table of Contents (rev. May 16, 2011)

%\usepackage{multicol}
%\usepackage{ifthen}
%\usepackage[toc]{multitoc}

%% General settings (rev. January 19, 2015)

\usepackage{ulem}

\usepackage[latin,english]{babel}
\usepackage{lettrine}

\usepackage{paracol}

\usepackage{fontspec}

\setmainfont[Ligatures=TeX,BoldFont=MinionPro-Bold,ItalicFont=MinionPro-It, BoldItalicFont=MinionPro-BoldIt]{MinionPro-Regular-Modified.otf}



%% Style for translation line
\grechangestyle{translation}{\fontsize{10}{10}\it\selectfont}
\grechangestyle{annotation}{\fontsize{10}{10}\selectfont}
\grechangestyle{commentary}{\textnormal\selectfont}
\gresetcustosalteration{invisible}

%\grechangedim{annotationseparation}{0.1cm}{scalable}

%\GreLoadSpaceConf{smith-four}

\frenchspacing

\usepackage{indentfirst} %%%indents first line after a section

\usepackage{graphicx}
%\usepackage{tocloft}

%%Hyperref (rev. August 20, 2011)
%\usepackage[colorlinks=false,hyperindex=true,bookmarks=true]{hyperref}
\usepackage{hyperref}
\hypersetup{pdftitle={Vesperale O.P. 2017}}
\hypersetup{pdfauthor={Order of Preachers}}
\hypersetup{pdfsubject={Liturgy}}
\hypersetup{pdfkeywords={Dominican, Liturgy, Order of Preachers, Dominican Rite, Liturgia Horarum, Divine Office}}

\newlength{\drop}



\begin{document}


%%%Initial Matter within Body (20 May 2011)
\raggedbottom

%%Combination

\chapter[Ss. Apostolorum Petri et Pauli]{Ss. Apostolorum Petri et Pauli}
\section{Ad II Vesperas}
    \index[Varia]{Deus in adiutorium} \label{Deus in adiutorium (Varia)} \grecommentary[0pt]{} \grechangestyle{initial}{\fontsize{1}{1}\selectfont} \grechangestyle{initial}{\fontsize{36}{36}\selectfont} \grechangedim{maxbaroffsettextleft@nobar}{12 cm}{scalable} \grechangedim{spaceabovelines}{0.5cm}{scalable} \gresetlyriccentering{vowel}  \grechangedim{maxbaroffsettextleft}{0 cm}{scalable} \gregorioscore{chants/misc.deus_in_adjutorium}   \vspace{5pt} \emph{}
 \subsection{Antiphona}  \greannotation{VIII \textsc{g}} \index[Antiphona]{Petrus et Ioannes} \label{Petrus et Ioannes (Antiphona)} \grecommentary[5pt]{} \grechangestyle{initial}{\fontsize{1}{1}\selectfont} \grechangestyle{initial}{\fontsize{36}{36}\selectfont} \grechangedim{maxbaroffsettextleft@nobar}{12 cm}{scalable} \grechangedim{spaceabovelines}{0.5cm}{scalable} \gresetlyriccentering{vowel}  \grechangedim{maxbaroffsettextleft}{0 cm}{scalable} \grechangedim{spaceabovelines}{0.5 cm}{scalable} \gregorioscore{chants/an--petrus_et_ioannes--dominican--id_6909}   \vspace{5pt} \emph{}
 \subsection{Psalm 109}   \index[Psalmus]{Psalm 109} \label{Psalm 109 (Psalmus)}         \gregorioscore{psalms-latin-vulgate/109-8G} \vspace{5pt} \par 2. Donec ponam inimícos t\uline{u}os,~* scabéllum ped\uuline{u}m tu\uline{ó}rum.\par 
3. Virgam virtútis tuæ emíttet Dóminus ex S\uline{i}on:~* domináre in médio inimicór\uuline{u}m tu\uline{ó}rum.\par 
4. Tecum princípium in die virtútis tuæ in splendóribus sanct\uline{ó}rum:~* ex útero ante lucíferum g\uuline{é}nu\uline{i} te.\par 
5. Jurávit Dóminus, et non pœnitébit \uline{e}um:~* Tu es sacérdos in ætérnum secúndum órdin\uuline{e}m Melch\uline{í}sedech.\par 
6. Dóminus a dextris t\uline{u}is,~* confrégit in die iræ s\uuline{u}æ r\uline{e}ges.\par 
7. Judicábit in natiónibus, implébit ru\uline{í}nas:~* conquassábit cápita in terr\uuline{a} mult\uline{ó}rum.\par 
8. De torrénte in via b\uline{i}bet:~* proptérea exalt\uuline{á}bit c\uline{a}put.\par 
9. Glória Patri, et F\uline{í}lio,~* et Spirít\uuline{u}i S\uline{a}ncto.\par 
10. Sicut erat in princípio, et nunc, et s\uline{e}mper,~* et in sǽcula sæcul\uuline{ó}rum. \uline{A}men.\par 

 \subsection{Psalm 112}   \index[Psalmus]{Psalm 112} \label{Psalm 112 (Psalmus)}         \gregorioscore{psalms-latin-vulgate/112-8G} \vspace{5pt} \par 2. Sit nomen Dómini bened\uline{í}ctum,~* ex hoc nunc, et us\uuline{que} in s\uline{ǽ}culum.
3. A solis ortu usque ad occ\uline{á}sum,~* laudábile n\uuline{o}men D\uline{ó}mini.
4. Excélsus super omnes gentes D\uline{ó}minus,~* et super cælos glór\uuline{i}a \uline{e}jus.
5. Quis sicut Dóminus, Deus noster, qui in altis h\uline{á}bitat,~* et humília réspicit in cælo \uuline{e}t in t\uline{e}rra?
6. Súscitans a terra \uline{í}nopem,~* et de stércore ér\uuline{i}gens p\uline{áu}perem:
7. Ut cóllocet eum cum princ\uline{í}pibus,~* cum princípibus póp\uuline{u}li s\uline{u}i.
8. Qui habitáre facit stérilem in d\uline{o}mo,~* matrem filiór\uuline{u}m læt\uline{á}ntem.
9. Glória Patri, et F\uline{í}lio,~* et Spirít\uuline{u}i S\uline{a}ncto.
10. Sicut erat in princípio, et nunc, et s\uline{e}mper,~* et in sǽcula sæcul\uuline{ó}rum. \uline{A}men.

 \subsection{Psalm 115}   \index[Psalmus]{Psalm 115} \label{Psalm 115 (Psalmus)}         \gregorioscore{psalms-latin-vulgate/115-8G} \vspace{5pt} \par 2. Ego dixi in excéssu m\uline{e}o:~* Omnis h\uuline{o}mo m\uline{e}ndax.
3. Quid retríbuam D\uline{ó}mino,~* pro ómnibus, quæ retríb\uuline{u}it m\uline{i}hi?
4. Cálicem salutáris acc\uline{í}piam:~* et nomen Dómini \uuline{i}nvoc\uline{á}bo.
5. Vota mea Dómino reddam coram omni pópulo \uline{e}jus:~* pretiósa in conspéctu Dómini mors sanct\uuline{ó}rum \uline{e}jus:
6. O Dómine, quia ego servus t\uline{u}us:~* ego servus tuus, et fílius anc\uuline{í}llæ t\uline{u}æ.
7. Dirupísti víncula mea:~† tibi sacrificábo hóstiam l\uline{au}dis,~* et nomen Dómini \uuline{i}nvoc\uline{á}bo.
8. Vota mea Dómino reddam in conspéctu omnis pópuli \uline{e}jus:~* in átriis domus Dómini, in médio tu\uuline{i}, Jer\uline{ú}salem.
9. Glória Patri, et F\uline{í}lio,~* et Spirít\uuline{u}i S\uline{a}ncto.
10. Sicut erat in princípio, et nunc, et s\uline{e}mper,~* et in sǽcula sæcul\uuline{ó}rum. \uline{A}men.

 \subsection{Psalm 125}   \index[Psalmus]{Psalm 125} \label{Psalm 125 (Psalmus)}         \gregorioscore{psalms-latin-vulgate/125-8G} \vspace{5pt} \par 2. Tunc replétum est gáudio os n\uline{o}strum:~* et lingua nostra exsult\uuline{a}ti\uline{ó}ne.
3. Tunc dicent inter G\uline{e}ntes:~* Magnificávit Dóminus fácer\uuline{e} cum \uline{e}is.
4. Magnificávit Dóminus fácere nob\uline{í}scum:~* facti sum\uuline{u}s læt\uline{á}ntes.
5. Convérte, Dómine, captivitátem n\uline{o}stram,~* sicut torr\uuline{e}ns in \uline{au}stro.
6. Qui séminant in l\uline{á}crimis,~* in exsultati\uuline{ó}ne m\uline{e}tent.
7. Eúntes ibant et fl\uline{e}bant,~* mitténtes sém\uuline{i}na s\uline{u}a.
8. Veniéntes autem vénient cum exsultati\uline{ó}ne,~* portántes maníp\uuline{u}los s\uline{u}os.
9. Glória Patri, et F\uline{í}lio,~* et Spirít\uuline{u}i S\uline{a}ncto.
10. Sicut erat in princípio, et nunc, et s\uline{e}mper,~* et in sǽcula sæcul\uuline{ó}rum. \uline{A}men.

 \subsection{Psalm 138}   \index[Psalmus]{Psalm 138} \label{Psalm 138 (Psalmus)}         \gregorioscore{psalms-latin-vulgate/138-8G} \vspace{5pt} \par 2. Intellexísti cogitatiónes meas de l\uline{o}nge:~* sémitam meam, et funículum meum inv\uuline{e}stig\uline{á}sti.
3. Et omnes vias meas prævid\uline{í}sti:~* quia non est sermo in l\uuline{i}ngua m\uline{e}a.
4. Ecce, Dómine, tu cognovísti ómnia novíssima, et ant\uline{í}qua:~* tu formásti me, et posuísti super me m\uuline{a}num t\uline{u}am.
5. Mirábilis facta est sciéntia tua \uline{e}x me:~* confortáta est, et non póter\uuline{o} ad \uline{e}am.
6. Quo ibo a spíritu t\uline{u}o?~* et quo a fácie t\uuline{u}a f\uline{ú}giam?
7. Si ascéndero in cælum, tu \uline{i}llic es:~* si descéndero in inf\uuline{é}rnum, \uline{a}des.
8. Si súmpsero pennas meas dil\uline{ú}culo,~* et habitávero in extr\uuline{é}mis m\uline{a}ris.
9. Etenim illuc manus tua ded\uline{ú}cet me:~* et tenébit me déxt\uuline{e}ra t\uline{u}a.
10. Et dixi: Fórsitan ténebræ conculc\uline{á}bunt me:~* et nox illuminátio mea in delíc\uuline{i}is m\uline{e}is.
11. Quia ténebræ non obscurabúntur a te,~† et nox sicut dies illumin\uline{á}bitur:~* sicut ténebræ ejus, ita et l\uuline{u}men \uline{e}jus.
12. Quia tu possedísti renes m\uline{e}os:~* suscepísti me de útero m\uuline{a}tris m\uline{e}æ.
13. Confitébor tibi quia terribíliter magnificátus es:~† mirabília ópera t\uline{u}a,~* et ánima mea cogn\uuline{ó}scit n\uline{i}mis.
14. Non est occultátum os meum a te, quod fecísti in occ\uline{ú}lto:~* et substántia mea in inferiór\uuline{i}bus t\uline{e}rræ.
15. Imperféctum meum vidérunt óculi tui,~† et in libro tuo omnes scrib\uline{é}ntur:~* dies formabúntur, et nem\uuline{o} in \uline{e}is.
16. Mihi autem nimis honorificáti sunt amíci tui, D\uline{e}us:~* nimis confortátus est principát\uuline{u}s e\uline{ó}rum.
17. Dinumerábo eos, et super arénam multiplicab\uline{ú}ntur:~* exsurréxi, et adh\uuline{u}c sum t\uline{e}cum.
18. Si occíderis, Deus, peccat\uline{ó}res:~* viri sánguinum, declin\uuline{á}te \uline{a} me.
19. Quia dícitis in cogitati\uline{ó}ne:~* Accípient in vanitáte civit\uuline{á}tes t\uline{u}as.
20. Nonne qui odérunt te, Dómine, \uline{ó}deram?~* et super inimícos tuos t\uuline{a}besc\uline{é}bam?
21. Perfécto ódio óderam \uline{i}llos:~* et inimíci fact\uuline{i} sunt m\uline{i}hi.
22. Proba me, Deus, et scito cor m\uline{e}um:~* intérroga me, et cognósce sém\uuline{i}tas m\uline{e}as.
23. Et vide, si via iniquitátis in m\uline{e} est:~* et deduc me in vi\uuline{a} æt\uline{é}rna.
24. Glória Patri, et F\uline{í}lio,~* et Spirít\uuline{u}i S\uline{a}ncto.
25. Sicut erat in princípio, et nunc, et s\uline{e}mper,~* et in sǽcula sæcul\uuline{ó}rum. \uline{A}men.

 \subsection{Antiphona}  \greannotation{VIII \textsc{g}} \index[Antiphona]{Petrus et Ioannes} \label{Petrus et Ioannes (Antiphona)} \grecommentary[5pt]{} \grechangestyle{initial}{\fontsize{1}{1}\selectfont} \grechangestyle{initial}{\fontsize{36}{36}\selectfont} \grechangedim{maxbaroffsettextleft@nobar}{12 cm}{scalable} \grechangedim{spaceabovelines}{0.5cm}{scalable} \gresetlyriccentering{vowel}  \grechangedim{maxbaroffsettextleft}{0 cm}{scalable} \grechangedim{spaceabovelines}{0.5 cm}{scalable} \gregorioscore{chants/an--petrus_et_ioannes--dominican--id_6909}   \vspace{5pt} \emph{}
 \subsection{Capitulum}     \hfill Eph 2        \lettrine[lines=3]{I}{}am non estis hóspites et ádvenæ, sed estis cives sanctórum et doméstici Dei, superædificáti super fundaméntum Apostolórum et Pro\-phe\-tár\-um, ipso summo angulári lápide Christo Iesu. \Rbar. Deo grátias.

 \subsection{Hymnus}  \greannotation{I} \index[Hymnus]{Aurea luce} \label{Aurea luce (Hymnus)} \grecommentary[10pt]{} \grechangestyle{initial}{\fontsize{1}{1}\selectfont} \grechangestyle{initial}{\fontsize{36}{36}\selectfont} \grechangedim{maxbaroffsettextleft@nobar}{12 cm}{scalable} \grechangedim{spaceabovelines}{0.5cm}{scalable} \gresetlyriccentering{vowel}  \grechangedim{spaceabovelines}{0.25 cm}{scalable} \gregorioscore{chants/hy--aurea_luce--latin}
 \subsection{Versiclua}             \par \Vbar. In omnem terram exívit sonus eórum. \par \Rbar. Et in fines orbis terræ verba eórum.
 \subsection{Antiphona ad Magnificat}  \greannotation{VI} \index[Antiphona ad Magnificat]{Gloriosi Principes} \label{Gloriosi Principes (Antiphona ad Magnificat)} \grecommentary[3pt]{} \grechangestyle{initial}{\fontsize{1}{1}\selectfont} \grechangestyle{initial}{\fontsize{36}{36}\selectfont} \grechangedim{maxbaroffsettextleft@nobar}{12 cm}{scalable} \grechangedim{spaceabovelines}{0.5cm}{scalable} \gresetlyriccentering{vowel}   \gregorioscore{chants/an--gloriosi_principes--dominican--id_6756}   \vspace{5pt} \emph{}
 \subsection{Canticum Evangelicum} \subsubsection{} \greannotation{VI} \index[Canticum Evangelicum]{Magnificat 6} \label{Magnificat 6 (Canticum Evangelicum)} \grecommentary[3pt]{Lc 1:46-55} \grechangestyle{initial}{\fontsize{1}{1}\selectfont} \grechangestyle{initial}{\fontsize{36}{36}\selectfont} \grechangedim{maxbaroffsettextleft@nobar}{12 cm}{scalable} \grechangedim{spaceabovelines}{0.5cm}{scalable} \gresetlyriccentering{vowel}  \grechangedim{spaceabovelines}{0.4 cm}{scalable} \gregorioscore{chants/magnificat6}  \grechangedim{spaceabovelines}{0 cm}{scalable}
 \subsection{Oratio}             \par \Vbar. Dóminus vobíscum. \par \Rbar. Et cum Spirítui tuo. \par Oremus. \par \lettrine[lines=3]{D}{}eus, qui hodiérnam diem Apostolórum tuórum Petri et Pauli martyrio consecrásti: da Ecclésiæ tuæ eórum in ómnibus sequi præcéptum, per quos religiónis sumpsit exordium. Per Dóminum nostrum Iesum Christum, Fílium tuum: qui tecum vivit et regnat in unitáte Spíritus Sancti Deus, per ómnia sǽcula sæculórum. \par \Rbar. Amen.
 \subsection{Benedicamus Domino}   \index[Benedicamus Domino]{Solemnities} \label{Solemnities (Benedicamus Domino)}         \gregorioscore{chants/misc.benedicamus.dominio.1}

  \end{document}
