\documentclass[11pt,twoside]{book}

%%Page Size (rev. 08/19/2016)
%\usepackage[inner=0.5in, outer=0.5in, top=0.5in, bottom=0.5in, papersize={6in,9in}, head=12pt, headheight=30pt, headsep=5pt]{geometry}
\usepackage[inner=0.5in, outer=0.5in, top=0.5in, bottom=0.5in, papersize={5.5in,8.5in}, head=12pt, headheight=30pt, headsep=5pt]{geometry}
%% width of textblock = 324 pt / 4.5in
%% A5 = 5.8 x 8.3 inches -- if papersize is A5, then margins should be [inner=0.75in, outer=0.55in, top=0.4in, bottom=0.4in]


%%Header (rev. 4/11/2011)
\usepackage{fancyhdr}
 \pagestyle{fancy}
\renewcommand{\chaptermark}[1]{\markboth{#1}{}}
\renewcommand{\sectionmark}[1]{\markright{#1}}
 \fancyhf{}
\fancyhead[LE,RO]{\thepage}
\fancyhead[CE]{\leftmark}
\fancyhead[CO]{\rightmark}
 \fancypagestyle{plain}{ %
\fancyhf{} % remove everything
\renewcommand{\headrulewidth}{0pt} % remove lines as well
\renewcommand{\footrulewidth}{0pt}}



\usepackage[autocompile,allowdeprecated=false]{gregoriotex}
\usepackage{gregoriosyms}
\gresetgregoriofont[op]{greciliae}




%%Titles (rev. 9/4/2011) -- TOCLESS --- lets you have sections that don't appear in the table of contents

\setcounter{secnumdepth}{-1}

\usepackage[compact,nobottomtitles*]{titlesec}
\titlespacing*{\chapter}{0pt}{-30pt}{0pt}
\titlespacing*{\section}{0pt}{*0}{*1}
\titlespacing*{\subsection}{0pt}{*0}{*1}
\titlespacing*{\subsubsection}{0pt}{*0}{*1}
\titleformat{\chapter} {\normalfont\LARGE\sc\center}{\thechapter}{0pt}{}
\titleformat{\section} {\normalfont\Large\sc\center}{\thesection}{1em}{}
\titleformat{\subsection} {\normalfont\large\sc\center}{\thesubsection}{1em}{}
\titleformat{\subsubsection}{\normalfont\normalsize\sc\center}{\thesubsubsection}{1em}{}

\newcommand{\nocontentsline}[3]{}
\newcommand{\tocless}[2]{\bgroup\let\addcontentsline=\nocontentsline#1{#2}\egroup} %% lets you have sections that don't appear in the table of contents


%%%

%%Index (rev. December 11, 2013)
\usepackage[noautomatic,nonewpage]{imakeidx}


\makeindex[name=incipit,title=Index]
\indexsetup{level=\section,toclevel=section,noclearpage}

\usepackage[indentunit=8pt,rule=.5pt,columns=2]{idxlayout}


%%Table of Contents (rev. May 16, 2011)

%\usepackage{multicol}
%\usepackage{ifthen}
%\usepackage[toc]{multitoc}

%% General settings (rev. January 19, 2015)

\usepackage{ulem}

\usepackage[latin,english]{babel}
\usepackage{lettrine}

\usepackage{paracol}

\usepackage{fontspec}

\setmainfont[Ligatures=TeX,BoldFont=MinionPro-Bold,ItalicFont=MinionPro-It, BoldItalicFont=MinionPro-BoldIt]{MinionPro-Regular-Modified.otf}



%% Style for translation line
\grechangestyle{translation}{\fontsize{10}{10}\it\selectfont}
\grechangestyle{annotation}{\fontsize{10}{10}\selectfont}
\grechangestyle{commentary}{\textnormal\selectfont}
\gresetcustosalteration{invisible}

%\grechangedim{annotationseparation}{0.1cm}{scalable}

%\GreLoadSpaceConf{smith-four}

\frenchspacing

\usepackage{indentfirst} %%%indents first line after a section

\usepackage{graphicx}
%\usepackage{tocloft}

%%Hyperref (rev. August 20, 2011)
%\usepackage[colorlinks=false,hyperindex=true,bookmarks=true]{hyperref}
\usepackage{hyperref}
\hypersetup{pdftitle={Vesperale O.P. 2016}}
\hypersetup{pdfauthor={Order of Preachers}}
\hypersetup{pdfsubject={Liturgy}}
\hypersetup{pdfkeywords={Dominican, Liturgy, Order of Preachers, Dominican Rite, Liturgia Horarum, Divine Office}}

\newlength{\drop}



\begin{document}


%%%Initial Matter within Body (20 May 2011)
\raggedbottom

%%Combination
\chapter{The Epiphany of the Lord}              
\section{Second Vespers}             \cleardoublepage 
    \index[Varia]{Deus in adiutorium} \label{Deus in adiutorium (Varia)} \grecommentary[0pt]{} \gresetinitiallines{1}  \grechangedim{maxbaroffsettextleft}{0 cm}{scalable} \gregorioscore{chants/misc.deus_in_adjutorium-T}  \grechangedim{maxbaroffsettextleft}{0.6 cm}{scalable}  
 \subsection{Hymnus}  \greannotation{III} \index[Hymnus]{Hostis Herodes impie} \label{Hostis Herodes impie (Hymnus)} \grecommentary[0pt]{} \gresetinitiallines{1} \gresetlyriccentering{syllable}  \gregorioscore{chants/hy--hostis_herodes--english}    
 \subsection{Antiphona}  \greannotation{VII a} \index[Antiphona]{Magnificatus est} \label{Magnificatus est (Antiphona)} \grecommentary[0pt]{1 Reg 10:23} \gresetinitiallines{1} \gresetlyriccentering{vowel} \grechangedim{maxbaroffsettextleft}{0 cm}{scalable} \gregorioscore{chants/an--magnificatus_est--dominican--id_6584}  \grechangedim{maxbaroffsettextleft}{0.6 cm}{scalable}  
 \subsection{Psalm 109} \subsubsection{The Messiah, king and priest}  \index[Psalmus]{Psalm 109} \label{Psalm 109 (Psalmus)} \emph{Christ’s reign will last until all his enemies are made subject to him (1~Cor 15:25).}    \vspace{5pt} \par %%underlines for psalm tones with three movements in the second and three movements in the third sections.

\noindent The Lord’s revelation to my \uline{Mas}ter:~†~\nopagebreak

“Sit \uline{on} my right:~$\star$~\nopagebreak

your foes I will put be\uline{neath} your feet.”

\noindent The Lord will wield from \uline{Zi}on~†~\nopagebreak

your scep\uline{ter} of power:~$\star$~\nopagebreak

rule in the midst of \uline{all} your foes.

\noindent A prince from the day of your \uline{birth}~†~\nopagebreak

on the \uline{ho}ly mountains;~$\star$~\nopagebreak

from the womb before the dawn \uline{I} begot you.

\noindent The Lord has sworn an oath he will not \uline{change}.~†~\nopagebreak

“You are a \uline{priest} for ever,~$\star$~\nopagebreak

a priest like Melchize\uline{dek} of old.”

\noindent The Master standing at \uline{your} right hand~$\star$~\nopagebreak

will shatter kings in the day of \uline{his} great wrath.

\noindent He shall drink from the stream \uline{by} the wayside~$\star$~\nopagebreak

and therefore he shall lift \uline{up} his head.

\noindent Glory to the Father, and \uline{to} the Son,~$\star$~\nopagebreak

and to the \uline{Ho}ly Spirit:

\noindent as it was in the begin\uline{ning}, is now,~$\star$~\nopagebreak

and will be for ev\uline{er}. Amen.
    
 \subsection{Antiphona}  \greannotation{VII d} \index[Antiphona]{Exortum est} \label{Exortum est (Antiphona)} \grecommentary[0pt]{Ps 111:4} \gresetinitiallines{1} \gresetlyriccentering{vowel} \grechangedim{maxbaroffsettextleft}{0 cm}{scalable} \gregorioscore{chants/an--exortum_est--dominican--id_5290}  \grechangedim{maxbaroffsettextleft}{0.6 cm}{scalable}  
 \subsection{Psalm 111} \subsubsection{Praise of God’s majesty}  \index[Psalmus]{Psalm 111} \label{Psalm 111 (Psalmus)} \emph{Lord, you are the Just One, who was and who is (Revelation 16:5).}    \vspace{5pt} \par \noindent Happy the man who \uline{fears} the Lord, ~$\star$~\nopagebreak

who takes delight in all \uline{his} commands.

\noindent His sons will be power\uline{ful} on earth; ~$\star$~\nopagebreak

the children of the up\uline{right} are blessed.

\noindent Riches and wealth are \uline{in} his house; ~$\star$~\nopagebreak

his justice stands \uline{firm} for ever.

\noindent He is a light in the darkness \uline{for} the upright: ~$\star$~\nopagebreak

he is generous, merci\uline{ful} and just.

\noindent The good man takes pi\uline{ty} and lends, ~$\star$~\nopagebreak

he conducts his af\uline{fairs} with honor.

\noindent The just man will \uline{nev}er waver: ~$\star$~\nopagebreak

he will be remem\uline{bered} for ever.

\noindent He has no fear of \uline{ev}il news; ~$\star$~\nopagebreak

with a firm heart he trusts \uline{in} the Lord.

\noindent With a steadfast heart he \uline{will} not fear; ~$\star$~\nopagebreak

he will see the downfall \uline{of} his foes.

\noindent Open-handed, he gives to the \uline{poor;} ~†~\nopagebreak

his justice stands \uline{firm} for ever. ~$\star$~\nopagebreak

His head will be \uline{raised} in glory.

\noindent The wicked man sees and is \uline{angry,} ~†~\nopagebreak

grinds his teeth and \uline{fades} away; ~$\star$~\nopagebreak

the desire of the wicked \uline{leads} to doom.

\noindent Glory to the Father, and \uline{to} the Son,~$\star$~\nopagebreak

and to the \uline{Ho}ly Spirit:

\noindent as it was in the begin\uline{ning}, is now,~$\star$~\nopagebreak

and will be for ev\uline{er}. Amen.
   \newpage 
 \subsection{Antiphona}  \greannotation{IV \textsc{e}} \index[Antiphona]{Omnes gentes} \label{Omnes gentes (Antiphona)} \grecommentary[0pt]{Ps 85:9} \gresetinitiallines{1}  \grechangedim{maxbaroffsettextleft}{0 cm}{scalable} \gregorioscore{chants/an--omnes_gentes_quascumque--dominican}  \grechangedim{maxbaroffsettextleft}{0.6 cm}{scalable}  
 \subsection{Ap 15:3-4} \subsubsection{Hymn of adoration}  \index[Canticum Evangelicum]{Ap 15:3-4} \label{Ap 15:3-4 (Canticum Evangelicum)}     \vspace{5pt} \par \noindent Mighty and wonderful \uline{are} your works,~$\star$~\nopagebreak

Lord \uline{God} Almighty!

\noindent Righteous and true \uline{are} your ways,~$\star$~\nopagebreak

O King \uline{of} the nations!

\noindent Who would dare re\uline{fuse} you honor,~$\star$~\nopagebreak

or the glory due your \uline{name}, O Lord?

\noindent Since you alone are \uline{holy},~†~\nopagebreak

all nations shall come and worship \uline{in} your presence.~$\star$~\nopagebreak

Your mighty deeds are \uline{clear}ly seen.

\noindent Glory to the Father, and \uline{to} the Son,~$\star$~\nopagebreak

and to the \uline{Ho}ly Spirit:

\noindent as it was in the begin\uline{ning}, is now,~$\star$~\nopagebreak

and will be for ev\uline{er}. Amen.   \newpage 
 \subsection{Lectio brevis}     \hfill Tit 3:4-5    \lettrine[lines=3]{B}{}ut when the kindness and generous love of God our savior appeared, not because of any righteous deeds we had done but because of his mercy, he saved us through the bath of rebirth and renewal by the Holy Spirit.    
 \subsection{Responsorium brevis}  \greannotation{VI} \index[Responsorium brevis]{Benedicentur in ipso} \label{Benedicentur in ipso (Responsorium brevis)} \grecommentary[0pt]{Ps 71:17 [AR]} \gresetinitiallines{1} \gresetlyriccentering{vowel}  \gregorioscore{chants/rb--benedicentur_in_ipso--solesmes}   \newpage 
 \subsection{Antiphona ad Magnificat}  \greannotation{I \textsc{d}} \index[Antiphona ad Magnificat]{Data est mihi} \label{Data est mihi (Antiphona ad Magnificat)} \grecommentary[0pt]{Ecclesia [AR]} \gresetinitiallines{1} \gresetlyriccentering{vowel}  \gregorioscore{chants/an--tribus_miraculis--solesmes2009} \vspace{5pt} \emph{Three miracles adorn this holy day we celebrate: today the star leads the Magi to the crib; today wine is made from water at the wedding feast; today Christ willed to be baptized by John, that he might save us, alleluia.}   
 \subsection{Canticum Evangelicum} \subsubsection{The soul rejoices in the Lord} \greannotation{I \textsc{d}} \index[Canticum Evangelicum]{Magnificat 8G} \label{Magnificat 8G (Canticum Evangelicum)} \grecommentary[0pt]{Lc 1:46-55} \gresetinitiallines{1} \gresetlyriccentering{vowel}  \gregorioscore{chants/magnificat1D} \vspace{10pt} 

\emph{My soul proclaims the greatness of the Lord, my spirit rejoices in God my Savior for he has looked with favor on his lowly servant. From this day all generations will call me blessed: the Almighty has done great things for me, and holy is his Name. He has mercy on those who fear him in every generation. He has shown the strength of his arm, he has scattered the proud in their conceit. He has cast down the mighty from their thrones, and has lifted up the lowly. He has filled the hungry with good things, and the rich he has sent away empty. He has come to the help of his servant Israel for he has remembered his promise of mercy, the promise he made to our fathers, to Abraham and his children for ever. Glory to the Father, and to the Son, and to the Holy Spirit: as it was in the beginning, is now, and will be for ever. Amen.}  \newpage 
 \subsection{Preces}   \index[Preces]{Epiphany} \label{Epiphany (Preces)}     \lettrine[lines=2]{T}{}oday, our Savior was adored by the Magi. Let us also worship him with joy as we pray: Save the poor, O Lord.
\par \Rbar. Save the poor, O Lord.

King of the nations, you called the Magi to adore you as the first representatives of the nations,
– give us a willing spirit of adoration and service.
\par \Rbar. Save the poor, O Lord.

King of glory, you judge your people with justice,
– grant mankind an abundant measure of peace.
\par \Rbar. Save the poor, O Lord.

King of ages, you endure from age to age,
– send your word as fresh spring rain falling on our hearts.
\par \Rbar. Save the poor, O Lord.

King of justice, you desire to free the poor who have no advocate,
– be compassionate to the suffering and the afflicted.
\par \Rbar. Save the poor, O Lord.

Lord, your name is blessed for ages,
– show the wonders of your saving power to our deceased brothers and sisters.
\par \Rbar. Save the poor, O Lord.   \newpage 
 \subsection{Pater noster}   \index[Pater noster]{Pater noster} \label{Pater noster (Pater noster)}     \gregorioscore{chants/or--pater_noster_a--solesmes-T}    
 \subsection{Oratio conclusiva}   \index[Oratio conclusiva]{Our Lord Jesus Christ, King of the Universe} \label{Our Lord Jesus Christ, King of the Universe (Oratio conclusiva)}     \lettrine[lines=3]{O}{} God, who on this day
revealed your Only Begotten Son to the nations
by the guidance of a star,
grant in your mercy
that we, who know you already by faith,
may be brought to behold the beauty of your sublime glory.
Through our Lord Jesus Christ, your Son,
who lives and reigns with you in the unity of the Holy Spirit,
one God, for ever and ever. \par \Rbar. Amen.     
 \subsection{Ritus conclusionis}         \par \Vbar. The Lord be with you. \par \Rbar. And with your spirit. \par \Vbar. May almighty God bless you, the Father, and the Son, and the Holy Spirit. \par \Rbar. Amen.    
 \subsection{Benedicamus Domino}   \index[Benedicamus Domino]{Solemnities} \label{Solemnities (Benedicamus Domino)}     \gregorioscore{chants/misc.benedicamus.dominio.1-T}       


  \end{document}
