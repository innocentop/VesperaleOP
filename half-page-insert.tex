\documentclass[11pt,twoside]{book}

%%Page Size (rev. 08/19/2016)
%\usepackage[inner=0.5in, outer=0.5in, top=0.5in, bottom=0.5in, papersize={6in,9in}, head=12pt, headheight=30pt, headsep=5pt]{geometry}
\usepackage[inner=0.5in, outer=0.5in, top=0.5in, bottom=0.5in, papersize={5.5in,8.5in}, head=12pt, headheight=30pt, headsep=5pt]{geometry}
%% width of textblock = 324 pt / 4.5in
%% A5 = 5.8 x 8.3 inches -- if papersize is A5, then margins should be [inner=0.75in, outer=0.55in, top=0.4in, bottom=0.4in]


%%Header (rev. 4/11/2011)
\usepackage{fancyhdr}
 \pagestyle{fancy}
\renewcommand{\chaptermark}[1]{\markboth{#1}{}}
\renewcommand{\sectionmark}[1]{\markright{#1}}
 \fancyhf{}
\fancyhead[LE,RO]{\thepage}
\fancyhead[CE]{\leftmark}
\fancyhead[CO]{\rightmark}
 \fancypagestyle{plain}{ %
\fancyhf{} % remove everything
\renewcommand{\headrulewidth}{0pt} % remove lines as well
\renewcommand{\footrulewidth}{0pt}}



\usepackage[autocompile,allowdeprecated=false]{gregoriotex}
\usepackage{gregoriosyms}
\gresetgregoriofont[op]{greciliae}




%%Titles (rev. 9/4/2011) -- TOCLESS --- lets you have sections that don't appear in the table of contents

\setcounter{secnumdepth}{-1}

\usepackage[compact,nobottomtitles*]{titlesec}
\titlespacing*{\chapter}{0pt}{-30pt}{0pt}
\titlespacing*{\section}{0pt}{*0}{*1}
\titlespacing*{\subsection}{0pt}{*0}{*1}
\titlespacing*{\subsubsection}{0pt}{*0}{*1}
\titleformat{\chapter} {\normalfont\LARGE\sc\center}{\thechapter}{0pt}{}
\titleformat{\section} {\normalfont\Large\sc\center}{\thesection}{1em}{}
\titleformat{\subsection} {\normalfont\large\sc\center}{\thesubsection}{1em}{}
\titleformat{\subsubsection}{\normalfont\normalsize\sc\center}{\thesubsubsection}{1em}{}

\newcommand{\nocontentsline}[3]{}
\newcommand{\tocless}[2]{\bgroup\let\addcontentsline=\nocontentsline#1{#2}\egroup} %% lets you have sections that don't appear in the table of contents


%%%

%%Index (rev. December 11, 2013)
\usepackage[noautomatic,nonewpage]{imakeidx}


\makeindex[name=incipit,title=Index]
\indexsetup{level=\section,toclevel=section,noclearpage}

\usepackage[indentunit=8pt,rule=.5pt,columns=2]{idxlayout}


%%Table of Contents (rev. May 16, 2011)

%\usepackage{multicol}
%\usepackage{ifthen}
%\usepackage[toc]{multitoc}

%% General settings (rev. January 19, 2015)

\usepackage{ulem}

\usepackage[latin,english]{babel}
\usepackage{lettrine}

\usepackage{paracol}

\usepackage{fontspec}

\setmainfont[Ligatures=TeX,BoldFont=MinionPro-Bold,ItalicFont=MinionPro-It, BoldItalicFont=MinionPro-BoldIt]{MinionPro-Regular-Modified.otf}



%% Style for translation line
\grechangestyle{translation}{\fontsize{10}{10}\it\selectfont}
\grechangestyle{annotation}{\fontsize{10}{10}\selectfont}
\grechangestyle{commentary}{\textnormal\selectfont}
\gresetcustosalteration{invisible}

%\grechangedim{annotationseparation}{0.1cm}{scalable}

%\GreLoadSpaceConf{smith-four}

\frenchspacing

\usepackage{indentfirst} %%%indents first line after a section

\usepackage{graphicx}
%\usepackage{tocloft}

%%Hyperref (rev. August 20, 2011)
%\usepackage[colorlinks=false,hyperindex=true,bookmarks=true]{hyperref}
\usepackage{hyperref}
\hypersetup{pdftitle={Vesperale O.P. 2016}}
\hypersetup{pdfauthor={Order of Preachers}}
\hypersetup{pdfsubject={Liturgy}}
\hypersetup{pdfkeywords={Dominican, Liturgy, Order of Preachers, Dominican Rite, Liturgia Horarum, Divine Office}}

\newlength{\drop}



\begin{document}


%%%Initial Matter within Body (20 May 2011)
\raggedbottom

%%Combination

 \subsection{Antiphona}  \greannotation{VII a} \index[Antiphona]{Assumpsit Iesus Petrum} \label{Assumpsit Iesus Petrum (Antiphona)} \grecommentary[3pt]{Mt 17:1-2} \gresetinitiallines{1} \gresetlyriccentering{vowel} \grechangedim{maxbaroffsettextleft}{0 cm}{scalable} \grechangedim{spaceabovelines}{0.5 cm}{scalable} \gregorioscore{chants/an--assumpsit_iesum_petrum--dominican--id_4759} \grechangedim{spaceabovelines}{0 cm}{scalable} \grechangedim{maxbaroffsettextleft}{0.6 cm}{scalable} \vspace{5pt} \emph{Jesus took Peter, as well as James and his brother John, and led them apart to a high mountain: and he was transfigured in their presence.}
 \subsection{Psalm 118:105-112} \subsubsection{A meditation on God’s law}  \index[Psalmus]{Psalm 118:105-112} \label{Psalm 118:105-112 (Psalmus)} \emph{This is my commandment: that you should love one another (John 15:12).}    \vspace{5pt} \par \noindent Not to us, Lord, \uline{not} to us,~$\star$~\nopagebreak

but to your name \uline{give} the glory

\noindent for the sake of your love \uline{and} your truth,~$\star$~\nopagebreak

lest the heathen say: “Where \uline{is} their God?”



\noindent But our God is \uline{in} the heavens;~$\star$~\nopagebreak

he does whatev\uline{er} he wills.

\noindent Their idols are sil\uline{ver} and gold,~$\star$~\nopagebreak

the work of \uline{hu}man hands.



\noindent They have mouths but they \uline{can}not speak;~$\star$~\nopagebreak

they have eyes but they \uline{can}not see;

\noindent they have ears but they \uline{can}not hear;~$\star$~\nopagebreak

they have nostrils but they \uline{can}not smell.



\noindent With their hands they cannot \uline{feel};~†~\nopagebreak

with their feet they \uline{can}not walk.~$\star$~\nopagebreak

No sound comes \uline{from} their throats.



\noindent Their makers will come to \uline{be} like them~$\star$~\nopagebreak

and so will all who \uline{trust} in them.



\noindent Sons of Israel, trust \uline{in} the Lord;~$\star$~\nopagebreak

he is their help \uline{and} their shield.

\noindent Sons of Aaron, trust \uline{in} the Lord;~$\star$~\nopagebreak

he is their help \uline{and} their shield.



\noindent You who fear him, trust \uline{in} the Lord;~$\star$~\nopagebreak

he is their help \uline{and} their shield.

\noindent He remembers us, and he will \uline{bless} us;~†~\nopagebreak

he will bless the \uline{sons} of Israel.~$\star$~\nopagebreak

He will bless the \uline{sons} of Aaron.



\noindent The Lord will bless \uline{those} who fear him,~$\star$~\nopagebreak

the little no less \uline{than} the great:

\noindent to you may the \uline{Lord} grant increase,~$\star$~\nopagebreak

to you and \uline{all} your children.



\noindent May you be blessed \uline{by} the Lord,~$\star$~\nopagebreak

the maker of hea\uline{ven} and earth.

\noindent The heavens belong \uline{to} the Lord~$\star$~\nopagebreak

but the earth he has gi\uline{ven} to men.



\noindent The dead shall not \uline{praise} the Lord,~$\star$~\nopagebreak

nor those who go down in\uline{to} the silence.

\noindent But we who live \uline{bless} the Lord~$\star$~\nopagebreak

now and for ev\uline{er}. Amen.



\noindent Glory to the Father, and \uline{to} the Son,~$\star$~\nopagebreak

and to the \uline{Ho}ly Spirit:

\noindent as it was in the begin\uline{ning}, is now,~$\star$~\nopagebreak

and will be for ev\uline{er}. Amen.



  \end{document}
