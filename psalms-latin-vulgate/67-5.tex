2. Sicut déficit fumus, def\uline{í}ciant:~* sicut fluit cera a fácie ignis, sic péreant peccatóres a f\uline{á}cie D\uline{e}i.
3. Et justi epuléntur, et exsúltent in conspéctu D\uline{e}i:~* et delecténtur \uline{i}n læt\uline{í}tia.
4. Cantáte Deo, psalmum dícite nómini \uline{e}jus:~* iter fácite ei, qui ascéndit super occásum: Dóminus n\uline{o}men \uline{i}lli.
5. Exsultáte in conspéctu \uline{e}jus:~* turbabúntur a fácie ejus, patris orphanórum et júdicis v\uline{i}du\uline{á}rum.
6. Deus in loco sancto s\uline{u}o:~* Deus, qui inhabitáre facit uníus m\uline{o}ris in d\uline{o}mo:
7. Qui edúcit vinctos in fortit\uline{ú}dine,~* simíliter eos qui exásperant, qui hábitant \uline{i}n sep\uline{ú}lcris.
8. Deus, cum egrederéris in conspéctu pópuli t\uline{u}i,~* cum pertransíres \uline{i}n des\uline{é}rto:
9. Terra mota est, étenim cæli distillavérunt a fácie Dei S\uline{í}nai,~* a fácie D\uline{e}i \uline{I}sraël.
10. Plúviam voluntáriam segregábis, Deus, hereditáti t\uline{u}æ:~* et infirmáta est, tu vero perfec\uline{í}sti \uline{e}am.
11. Animália tua habitábunt in \uline{e}a:~* parásti in dulcédine tua p\uline{áu}peri, D\uline{e}us.
12. Dóminus dabit verbum evangeliz\uline{á}ntibus,~* virt\uline{ú}te m\uline{u}lta.
13. Rex virtútum dilécti dil\uline{é}cti:~* et speciéi domus div\uline{í}dere sp\uline{ó}lia.
14. Si dormiátis inter médios cleros, pennæ colúmbæ deargent\uline{á}tæ,~* et posterióra dorsi ejus in pall\uline{ó}re \uline{au}ri.
15. Dum discérnit cæléstis reges super eam, nive dealbabúntur in S\uline{e}lmon:~* mons D\uline{e}i, mons p\uline{i}nguis.
16. Mons coagulátus, mons p\uline{i}nguis:~* ut quid suspicámini montes co\uline{a}gul\uline{á}tos?
17. Mons, in quo beneplácitum est Deo habitáre in \uline{e}o:~* étenim Dóminus habit\uline{á}bit in f\uline{i}nem.
18. Currus Dei decem míllibus múltiplex, míllia læt\uline{á}ntium:~* Dóminus in eis in S\uline{i}na in s\uline{a}ncto.
19. Ascendísti in altum, cepísti captivit\uline{á}tem:~* accepísti dona \uline{i}n hom\uline{í}nibus.
20. Etenim non cred\uline{é}ntes,~* inhabitáre D\uline{ó}minum D\uline{e}um.
21. Benedíctus Dóminus die quot\uline{í}die:~* prósperum iter fáciet nobis Deus salutári\uline{u}m nostr\uline{ó}rum.
22. Deus noster, Deus salvos faci\uline{é}ndi:~* et Dómini Dómini \uline{é}xitus m\uline{o}rtis.
23. Verúmtamen Deus confrínget cápita inimicórum su\uline{ó}rum:~* vérticem capílli perambulántium in del\uline{í}ctis s\uline{u}is.
24. Dixit Dóminus: Ex Basan conv\uline{é}rtam,~* convértam in prof\uline{ú}ndum m\uline{a}ris:
25. Ut intingátur pes tuus in s\uline{á}nguine:~* lingua canum tuórum ex inim\uline{í}cis, ab \uline{i}pso.
26. Vidérunt ingréssus tuos, D\uline{e}us:~* ingréssus Dei mei: regis mei qui \uline{e}st in s\uline{a}ncto.
27. Prævenérunt príncipes conjúncti psall\uline{é}ntibus:~* in médio juvenculárum tympan\uline{i}stri\uline{á}rum.
28. In ecclésiis benedícite Deo D\uline{ó}mino,~* de f\uline{ó}ntibus \uline{I}sraël.
29. Ibi Bénjamin adolesc\uline{é}ntulus:~* in m\uline{e}ntis exc\uline{é}ssu.
30. Príncipes Juda, duces e\uline{ó}rum:~* príncipes Zábulon, pr\uline{í}ncipes N\uline{é}phtali.
31. Manda, Deus, virtúti t\uline{u}æ:~* confírma hoc, Deus, quod operátus \uline{e}s in n\uline{o}bis.
32. A templo tuo in Jer\uline{ú}salem,~* tibi ófferent r\uline{e}ges m\uline{ú}nera.
33. Increpa feras arúndinis, congregátio taurórum in vaccis popul\uline{ó}rum:~* ut exclúdant eos, qui probáti s\uline{u}nt arg\uline{é}nto.
34. Díssipa Gentes, quæ bella volunt: vénient legáti ex Æg\uline{ý}pto:~* Æthiópia prævéniet manus \uline{e}jus D\uline{e}o.
35. Regna terræ, cantáte D\uline{e}o:~* ps\uline{á}llite D\uline{ó}mino.
36. Psállite Deo, qui ascéndit super cælum c\uline{æ}li,~* ad \uline{O}ri\uline{é}ntem.
37. Ecce dabit voci suæ vocem virtútis, date glóriam Deo super \uline{I}sraël,~* magnificéntia ejus, et virtus \uline{e}jus in n\uline{ú}bibus.
38. Mirábilis Deus in sanctis suis, Deus Israël ipse dabit virtútem, et fortitúdinem plebi s\uline{u}æ,~* bened\uline{í}ctus D\uline{e}us.
39. Glória Patri, et F\uline{í}lio,~* et Spir\uline{í}tui S\uline{a}ncto.
40. Sicut erat in princípio, et nunc, et s\uline{e}mper,~* et in sǽcula sæcul\uline{ó}rum. \uline{A}men.
