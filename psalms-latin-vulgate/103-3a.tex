2. Confessiónem, et decórem \uline{i}ndu\uline{í}sti:~* amíctus lúmine sicut v\uline{e}stim\uline{é}nto.
3. Exténdens cælum s\uline{i}cut p\uline{e}llem:~* qui tegis aquis superi\uline{ó}ra \uline{e}jus.
4. Qui ponis nubem asc\uline{é}nsum t\uline{u}um:~* qui ámbulas super p\uline{e}nnas vent\uline{ó}rum.
5. Qui facis ángelos tuos, sp\uline{í}r\uline{i}tus:~* et minístros tuos \uline{i}gnem ur\uline{é}ntem.
6. Qui fundásti terram super stabilit\uline{á}tem s\uline{u}am:~* non inclinábitur in s\uline{ǽ}culum s\uline{ǽ}culi.
7. Abýssus, sicut vestiméntum, am\uline{í}ctus \uline{e}jus:~* super montes st\uline{a}bunt \uline{a}quæ.
8. Ab increpatióne tua f\uline{ú}g\uline{i}ent:~* a voce tonítrui tui f\uline{o}rmid\uline{á}bunt.
9. Ascéndunt montes: et desc\uline{é}ndunt c\uline{a}mpi~* in locum, quem fund\uline{á}sti \uline{e}is.
10. Términum posuísti, quem non transgr\uline{e}di\uline{é}ntur:~* neque converténtur oper\uline{í}re t\uline{e}rram.
11. Qui emíttis fontes in conv\uline{á}ll\uline{i}bus:~* inter médium móntium pertrans\uline{í}bunt \uline{a}quæ.
12. Potábunt omnes b\uline{é}stiæ \uline{a}gri:~* exspectábunt ónagri in s\uline{i}ti s\uline{u}a.
13. Super ea vólucres cæli h\uline{a}bit\uline{á}bunt:~* de médio petrárum d\uline{a}bunt v\uline{o}ces.
14. Rigans montes de superi\uline{ó}ribus s\uline{u}is:~* de fructu óperum tuórum sati\uline{á}bitur t\uline{e}rra:
15. Prodúcens f\uline{œ}num jum\uline{é}ntis:~* et herbam servit\uline{ú}ti h\uline{ó}minum:
16. Ut edúcas p\uline{a}nem de t\uline{e}rra:~* et vinum lætífic\uline{e}t cor h\uline{ó}minis:
17. Ut exhílaret fáciem in \uline{ó}l\uline{e}o:~* et panis cor hómin\uline{i}s conf\uline{í}rmet.
18. Saturabúntur ligna campi, et cedri Líbani, \uline{qua}s plant\uline{á}vit:~* illic pásseres nid\uline{i}fic\uline{á}bunt.
19. Heródii domus dux \uline{e}st e\uline{ó}rum:~* montes excélsi cervis: petra refúgium h\uline{e}rin\uline{á}ciis.
20. Fecit lunam in t\uline{é}mp\uline{o}ra:~* sol cognóvit occ\uline{á}sum s\uline{u}um.
21. Posuísti ténebras, et f\uline{a}cta \uline{e}st nox:~* in ipsa pertransíbunt omnes b\uline{é}stiæ s\uline{i}lvæ.
22. Cátuli leónum rugiéntes, ut r\uline{á}p\uline{i}ant:~* et quærant a Deo \uline{e}scam s\uline{i}bi.
23. Ortus est sol, et congreg\uline{á}t\uline{i} sunt:~* et in cubílibus suis coll\uline{o}cab\uline{ú}ntur.
24. Exíbit homo ad \uline{o}pus s\uline{u}um:~* et ad operatiónem suam \uline{u}sque ad v\uline{é}sperum.
25. Quam magnificáta sunt ópera tua, D\uline{ó}m\uline{i}ne!~* ómnia in sapiéntia fecísti: impléta est terra possessi\uline{ó}ne t\uline{u}a.
26. Hoc mare magnum, et spatiósum m\uline{á}n\uline{i}bus:~* illic reptília, quorum n\uline{o}n est n\uline{ú}merus.
27. Animália pus\uline{í}lla cum m\uline{a}gnis:~* illic naves p\uline{e}rtrans\uline{í}bunt.
28. Draco iste, quem formásti ad illud\uline{é}ndum \uline{e}i:~* ómnia a te exspéctant ut des illis \uline{e}scam in t\uline{é}mpore.
29. Dante te illis, c\uline{ó}ll\uline{i}gent:~* aperiénte te manum tuam, ómnia implebúntur b\uline{o}nit\uline{á}te.
30. Averténte autem te fáciem, t\uline{u}rbab\uline{ú}ntur:~* áuferes spíritum eórum, et defícient, et in púlverem suum r\uline{e}vert\uline{é}ntur.
31. Emíttes spíritum tuum, et cr\uline{e}ab\uline{ú}ntur:~* et renovábis f\uline{á}ciem t\uline{e}rræ.
32. Sit glória Dómini in s\uline{ǽ}c\uline{u}lum:~* lætábitur Dóminus in op\uline{é}ribus s\uline{u}is:
33. Qui réspicit terram, et facit eam tr\uline{é}m\uline{e}re:~* qui tangit m\uline{o}ntes, et f\uline{ú}migant.
34. Cantábo Dómino in v\uline{i}ta m\uline{e}a:~* psallam Deo meo, \uline{quá}mdi\uline{u} sum.
35. Jucúndum sit ei el\uline{ó}quium m\uline{e}um:~* ego vero delect\uline{á}bor in D\uline{ó}mino.
36. Defíciant peccatóres a terra, et iníqui \uline{i}ta ut n\uline{o}n sint:~* bénedic, ánima m\uline{e}a, D\uline{ó}mino.
37. Glória Patri, et F\uline{í}l\uline{i}o,~* et Spir\uline{í}tui S\uline{a}ncto.
38. Sicut erat in princípio, et n\uline{u}nc, et s\uline{e}mper,~* et in sǽcula sæcul\uline{ó}rum. \uline{A}men.
