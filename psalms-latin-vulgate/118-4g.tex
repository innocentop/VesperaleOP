2. Beáti, qui scrutántur testimón\uuline{i}a \uline{e}jus:~* in toto corde exquírunt \uline{e}um.
3. Non enim qui operántur in\uuline{i}quit\uline{á}tem,~* in viis ejus ambulav\uline{é}runt.
4. T\uuline{u} mand\uline{á}sti~* mandáta tua custodíri n\uline{i}mis.
5. Utinam dirigántur v\uuline{i}æ m\uline{e}æ,~* ad custodiéndas justificatiónes t\uline{u}as!
6. Tunc n\uuline{o}n conf\uline{ú}ndar,~* cum perspéxero in ómnibus mandátis t\uline{u}is.
7. Confitébor tibi in directi\uuline{ó}ne c\uline{o}rdis:~* in eo quod dídici judícia justítiæ t\uline{u}æ.
8. Justificatiónes tu\uuline{a}s cust\uline{ó}diam:~* non me derelínquas usque\uline{quá}que.
9. In quo córrigit adolescéntior v\uuline{i}am s\uline{u}am?~* in custodiéndo sermónes t\uline{u}os.
10. In toto corde meo \uuline{e}xquis\uline{í}vi te:~* ne repéllas me a mandátis t\uline{u}is.
11. In corde meo abscóndi eló\uuline{qui}a t\uline{u}a:~* ut non peccem t\uline{i}bi.
12. Benedíct\uuline{u}s es, D\uline{ó}mine:~* doce me justificatiónes t\uline{u}as.
13. In láb\uuline{i}is m\uline{e}is,~* pronuntiávi ómnia judícia oris t\uline{u}i.
14. In via testimoniórum tuórum d\uuline{e}lect\uline{á}tus sum,~* sicut in ómnibus div\uline{í}tiis.
15. In mandátis tuis \uuline{e}xerc\uline{é}bor:~* et considerábo vias t\uline{u}as.
16. In justificatiónibus tuis m\uuline{e}dit\uline{á}bor:~* non oblivíscar sermónes t\uline{u}os.
17. Retríbue servo tuo, viv\uuline{í}fic\uline{a} me:~* et custódiam sermónes t\uline{u}os.
18. Revéla óc\uuline{u}los m\uline{e}os:~* et considerábo mirabília de lege t\uline{u}a.
19. Incola ego s\uuline{u}m in t\uline{e}rra:~* non abscóndas a me mandáta t\uline{u}a.
20. Concupívit ánima mea desideráre justificati\uuline{ó}nes t\uline{u}as,~* in omni t\uline{é}mpore.
21. Increpást\uuline{i} sup\uline{é}rbos:~* maledícti qui declínant a mandátis t\uline{u}is.
22. Aufer a me oppróbrium, \uuline{e}t cont\uline{é}mptum:~* quia testimónia tua exquis\uline{í}vi.
23. Etenim sedérunt príncipes, et advérsum me l\uuline{o}queb\uline{á}ntur:~* servus autem tuus exercebátur in justificatiónibus t\uline{u}is.
24. Nam et testimónia tua meditát\uuline{i}o m\uline{e}a est:~* et consílium meum justificatiónes t\uline{u}æ.
25. Adhǽsit paviménto án\uuline{i}ma m\uline{e}a:~* vivífica me secúndum verbum t\uline{u}um.
26. Vias meas enuntiávi et \uuline{e}xaud\uline{í}sti me:~* doce me justificatiónes t\uline{u}as.
27. Viam justificatiónum tuárum \uuline{í}nstru\uline{e} me:~* et exercébor in mirabílibus t\uline{u}is.
28. Dormitávit ánima me\uuline{a} præ t\uline{ǽ}dio:~* confírma me in verbis t\uline{u}is.
29. Viam iniquitátis ám\uuline{o}ve \uline{a} me:~* et de lege tua miserére m\uline{e}i.
30. Viam veritát\uuline{i}s el\uline{é}gi:~* judícia tua non sum obl\uline{í}tus.
31. Adhǽsi testimóniis t\uuline{u}is D\uline{ó}mine:~* noli me conf\uline{ú}ndere.
32. Viam mandatórum tuór\uuline{u}m cuc\uline{ú}rri:~* cum dilatásti cor m\uline{e}um.
33. Legem pone mihi, Dómine, viam justificatión\uuline{u}m tu\uline{á}rum:~* et exquíram eam s\uline{e}mper.
34. Da mihi intelléctum, et scrutábor l\uuline{e}gem t\uline{u}am:~* et custódiam illam in toto corde m\uline{e}o.
35. Deduc me in sémitam mandatór\uuline{u}m tu\uline{ó}rum:~* quia ipsam v\uline{ó}lui.
36. Inclína cor meum in testimón\uuline{i}a t\uline{u}a:~* et non in avar\uline{í}tiam.
37. Avérte óculos meos ne vídeant v\uuline{a}nit\uline{á}tem:~* in via tua vivífic\uline{a} me.
38. Státue servo tuo eló\uuline{qui}um t\uline{u}um,~* in timóre t\uline{u}o.
39. Amputa oppróbrium meum quod s\uuline{u}spic\uline{á}tus sum:~* quia judícia tua juc\uline{ú}nda.
40. Ecce concupívi mand\uuline{á}ta t\uline{u}a:~* in æquitáte tua vivífic\uline{a} me.
41. Et véniat super me misericórdia t\uuline{u}a, D\uline{ó}mine:~* salutáre tuum secúndum elóquium t\uline{u}um.
42. Et respondébo exprobrántibus m\uuline{i}hi v\uline{e}rbum:~* quia sperávi in sermónibus t\uline{u}is.
43. Et ne áuferas de ore meo verbum veritátis \uuline{u}sque\uline{quá}que:~* quia in judíciis tuis supersper\uline{á}vi.
44. Et custódiam legem t\uuline{u}am s\uline{e}mper:~* in sǽculum et in sǽculum s\uline{ǽ}culi.
45. Et ambulábam in l\uuline{a}tit\uline{ú}dine:~* quia mandáta tua exquis\uline{í}vi.
46. Et loquébar in testimóniis tuis in consp\uuline{é}ctu r\uline{e}gum:~* et non confund\uline{é}bar.
47. Et meditábar in mand\uuline{á}tis t\uline{u}is,~* quæ dil\uline{é}xi.
48. Et levávi manus meas ad mandáta tua, \uuline{quæ} dil\uline{é}xi:~* et exercébar in justificatiónibus t\uline{u}is.
49. Memor esto verbi tui s\uuline{e}rvo t\uline{u}o,~* in quo mihi spem ded\uline{í}sti.
50. Hæc me consoláta est in humilit\uuline{á}te m\uline{e}a:~* quia elóquium tuum vivific\uline{á}vit me.
51. Supérbi iníque agébant \uuline{u}sque\uline{quá}que:~* a lege autem tua non declin\uline{á}vi.
52. Memor fui judiciórum tuórum a sǽc\uuline{u}lo, D\uline{ó}mine:~* et consol\uline{á}tus sum.
53. Deféctio t\uuline{é}nu\uline{i}t me,~* pro peccatóribus derelinquéntibus legem t\uline{u}am.
54. Cantábiles mihi erant justificati\uuline{ó}nes t\uline{u}æ,~* in loco peregrinatiónis m\uline{e}æ.
55. Memor fui nocte nóminis t\uuline{u}i, D\uline{ó}mine:~* et custodívi legem t\uline{u}am.
56. Hæc fact\uuline{a} est m\uline{i}hi:~* quia justificatiónes tuas exquis\uline{í}vi.
57. Pórtio m\uuline{e}a, D\uline{ó}mine,~* dixi custodíre legem t\uline{u}am.
58. Deprecátus sum fáciem tuam in toto c\uuline{o}rde m\uline{e}o:~* miserére mei secúndum elóquium t\uline{u}um.
59. Cogitávi v\uuline{i}as m\uline{e}as:~* et convérti pedes meos in testimónia t\uline{u}a.
60. Parátus sum, et non s\uuline{u}m turb\uline{á}tus:~* ut custódiam mandáta t\uline{u}a.
61. Funes peccatórum circumpl\uuline{é}xi s\uline{u}nt me:~* et legem tuam non sum obl\uline{í}tus.
62. Média nocte surgébam ad confit\uuline{é}ndum t\uline{i}bi:~* super judícia justificatiónis t\uline{u}æ.
63. Párticeps ego sum ómnium tim\uuline{é}nti\uline{u}m te:~* et custodiéntium mandáta t\uline{u}a.
64. Misericórdia tua, Dómine, plen\uuline{a} est t\uline{e}rra:~* justificatiónes tuas d\uline{o}ce me.
65. Bonitátem fecísti cum servo t\uuline{u}o, D\uline{ó}mine:~* secúndum verbum t\uline{u}um.
66. Bonitátem et disciplínam et sciént\uuline{i}am d\uline{o}ce me:~* quia mandátis tuis cr\uline{é}didi.
67. Priúsquam humiliárer eg\uuline{o} del\uline{í}qui:~* proptérea elóquium tuum custod\uline{í}vi.
68. B\uuline{o}nus \uline{e}s tu:~* et in bonitáte tua doce me justificatiónes t\uline{u}as.
69. Multiplicáta est super me iníquitas s\uuline{u}perb\uline{ó}rum:~* ego autem in toto corde meo scrutábor mandáta t\uline{u}a.
70. Coagulátum est sicut lac c\uuline{o}r e\uline{ó}rum:~* ego vero legem tuam medit\uline{á}tus sum.
71. Bonum mihi quia hum\uuline{i}li\uline{á}sti me:~* ut discam justificatiónes t\uline{u}as.
72. Bonum mihi lex \uuline{o}ris t\uline{u}i:~* super míllia auri et arg\uline{é}nti.
73. Manus tuæ fecérunt me, et pl\uuline{a}smav\uline{é}runt me:~* da mihi intelléctum, et discam mandáta t\uline{u}a.
74. Qui timent te vidébunt me et l\uuline{æ}tab\uline{ú}ntur:~* quia in verba tua supersper\uline{á}vi.
75. Cognóvi, Dómine, quia ǽquitas judíc\uuline{i}a t\uline{u}a:~* et in veritáte tua humili\uline{á}sti me.
76. Fiat misericórdia tua ut c\uuline{o}nsol\uline{é}tur me:~* secúndum elóquium tuum servo t\uline{u}o.
77. Véniant mihi miseratiónes tu\uuline{æ}, et v\uline{i}vam:~* quia lex tua meditátio m\uline{e}a est.
78. Confundántur supérbi, quia injúste iniquitátem fec\uuline{é}runt \uline{i}n me:~* ego autem exercébor in mandátis t\uline{u}is.
79. Convertántur mih\uuline{i} tim\uline{é}ntes te:~* et qui novérunt testimónia t\uline{u}a.
80. Fiat cor meum immaculátum in justificatión\uuline{i}bus t\uline{u}is,~* ut non conf\uline{ú}ndar.
81. Defécit in salutáre tuum án\uuline{i}ma m\uline{e}a:~* et in verbum tuum supersper\uline{á}vi.
82. Defecérunt óculi mei in eló\uuline{qui}um t\uline{u}um:~* dicéntes: Quando consoláber\uline{i}s me?
83. Quia factus sum sicut uter \uuline{i}n pru\uline{í}na:~* justificatiónes tuas non sum obl\uline{í}tus.
84. Quot sunt dies s\uuline{e}rvi t\uline{u}i?~* quando fácies de persequéntibus me jud\uline{í}cium?
85. Narravérunt mihi iníqui fabul\uuline{a}ti\uline{ó}nes:~* sed non ut lex t\uline{u}a.
86. Omnia mandáta t\uuline{u}a v\uline{é}ritas:~* iníque persecúti sunt me, ádjuv\uline{a} me.
87. Paulo minus consummavérunt m\uuline{e} in t\uline{e}rra:~* ego autem non derelíqui mandáta t\uline{u}a.
88. Secúndum misericórdiam tuam viv\uuline{í}fic\uline{a} me:~* et custódiam testimónia oris t\uline{u}i.
89. In ætérnum, Dómine, verbum tuum pérman\uuline{e}t in c\uline{æ}lo.
90. In generatiónem et generatiónem vér\uuline{i}tas t\uline{u}a:~* fundásti terram, et p\uline{é}rmanet.
91. Ordinatióne tua persev\uuline{é}rat d\uline{i}es:~* quóniam ómnia sérviunt t\uline{i}bi.
92. Nisi quod lex tua meditát\uuline{i}o m\uline{e}a est:~* tunc forte periíssem in humilitáte m\uline{e}a.
93. In ætérnum non oblivíscar justificati\uuline{ó}nes t\uline{u}as:~* quia in ipsis vivific\uline{á}sti me.
94. Tuus sum ego, s\uuline{a}lvum m\uline{e} fac:~* quóniam justificatiónes tuas exquis\uline{í}vi.
95. Me exspectavérunt peccatóres ut p\uuline{é}rder\uline{e}nt me:~* testimónia tua intell\uline{é}xi.
96. Omnis consummatiónis v\uuline{i}di f\uline{i}nem:~* latum mandátum tuum n\uline{i}mis.
97. Quómodo diléxi legem t\uuline{u}am, D\uline{ó}mine?~* tota die meditátio m\uline{e}a est.
98. Super inimícos meos prudéntem me fecísti mand\uuline{á}to t\uline{u}o:~* quia in ætérnum m\uline{i}hi est.
99. Super omnes docéntes me \uuline{i}ntell\uline{é}xi:~* quia testimónia tua meditátio m\uline{e}a est.
100. Super senes \uuline{i}ntell\uline{é}xi:~* quia mandáta tua quæs\uline{í}vi.
101. Ab omni via mala prohíbui p\uuline{e}des m\uline{e}os:~* ut custódiam verba t\uline{u}a.
102. A judíciis tuis non d\uuline{e}clin\uline{á}vi:~* quia tu legem posuísti m\uline{i}hi.
103. Quam dúlcia fáucibus meis eló\uuline{qui}a t\uline{u}a,~* super mel ori m\uline{e}o!
104. A mandátis tuis \uuline{i}ntell\uline{é}xi:~* proptérea odívi omnem viam iniquit\uline{á}tis.
105. Lucérna pédibus meis v\uuline{e}rbum t\uline{u}um,~* et lumen sémitis m\uline{e}is.
106. Juráv\uuline{i}, et st\uline{á}tui~* custodíre judícia justítiæ t\uline{u}æ.
107. Humiliátus sum usque\uuline{quá}que, D\uline{ó}mine:~* vivífica me secúndum verbum t\uline{u}um.
108. Voluntária oris mei beneplácit\uuline{a} fac, D\uline{ó}mine:~* et judícia tua d\uline{o}ce me.
109. Anima mea in mánibus m\uuline{e}is s\uline{e}mper:~* et legem tuam non sum obl\uline{í}tus.
110. Posuérunt peccatóres lá\uuline{que}um m\uline{i}hi:~* et de mandátis tuis non err\uline{á}vi.
111. Hereditáte acquisívi testimónia tua \uuline{i}n æt\uline{é}rnum:~* quia exsultátio cordis m\uline{e}i sunt.
112. Inclinávi cor meum ad faciéndas justificatiónes tuas \uuline{i}n æt\uline{é}rnum,~* propter retributi\uline{ó}nem.
113. Iníquos ód\uuline{i}o h\uline{á}bui:~* et legem tuam dil\uline{é}xi.
114. Adjútor et suscéptor m\uuline{e}us \uline{e}s tu:~* et in verbum tuum supersper\uline{á}vi.
115. Declináte a m\uuline{e}, mal\uline{í}gni:~* et scrutábor mandáta Dei m\uline{e}i.
116. Súscipe me secúndum elóquium tu\uuline{u}m, et v\uline{i}vam:~* et non confúndas me ab exspectatióne m\uline{e}a.
117. Adjuva me, et s\uuline{a}lvus \uline{e}ro:~* et meditábor in justificatiónibus tuis s\uline{e}mper.
118. Sprevísti omnes discedéntes a judíc\uuline{i}is t\uline{u}is:~* quia injústa cogitátio e\uline{ó}rum.
119. Prævaricántes reputávi omnes peccat\uuline{ó}res t\uline{e}rræ:~* ídeo diléxi testimónia t\uline{u}a.
120. Confíge timóre tuo c\uuline{a}rnes m\uline{e}as:~* a judíciis enim tuis t\uline{í}mui.
121. Feci judícium \uuline{e}t just\uline{í}tiam:~* non tradas me calumniántib\uline{u}s me.
122. Súscipe servum tu\uuline{u}m in b\uline{o}num:~* non calumniéntur me sup\uline{é}rbi.
123. Oculi mei defecérunt in salut\uuline{á}re t\uline{u}um:~* et in elóquium justítiæ t\uline{u}æ.
124. Fac cum servo tuo secúndum misericórd\uuline{i}am t\uline{u}am:~* et justificatiónes tuas d\uline{o}ce me.
125. Servus tu\uuline{u}s sum \uline{e}go:~* da mihi intelléctum, ut sciam testimónia t\uline{u}a.
126. Tempus faci\uuline{é}ndi, D\uline{ó}mine:~* díssipavérunt legem t\uline{u}am.
127. Ideo diléxi mand\uuline{á}ta t\uline{u}a,~* super aurum et top\uline{á}zion.
128. Proptérea ad ómnia mandáta tua d\uuline{i}rig\uline{é}bar:~* omnem viam iníquam ódio h\uline{á}bui.
129. Mirabília testimón\uuline{i}a t\uline{u}a:~* ídeo scrutáta est ea ánima m\uline{e}a.
130. Declarátio sermónum tuór\uuline{u}m ill\uline{ú}minat:~* et intelléctum dat p\uline{á}rvulis.
131. Os meum apérui, et attr\uuline{á}xi sp\uline{í}ritum:~* quia mandáta tua desider\uline{á}bam.
132. Aspice in me, et miser\uuline{é}re m\uline{e}i:~* secúndum judícium diligéntium nomen t\uline{u}um.
133. Gressus meos dírige secúndum eló\uuline{qui}um t\uline{u}um:~* et non dominétur mei omnis injust\uline{í}tia.
134. Rédime me a calúmn\uuline{i}is h\uline{ó}minum:~* ut custódiam mandáta t\uline{u}a.
135. Fáciem tuam illúmina super s\uuline{e}rvum t\uline{u}um:~* et doce me justificatiónes t\uline{u}as.
136. Exitus aquárum deduxérunt óc\uuline{u}li m\uline{e}i:~* quia non custodiérunt legem t\uline{u}am.
137. Just\uuline{u}s es, D\uline{ó}mine:~* et rectum judícium t\uline{u}um.
138. Mandásti justítiam testimón\uuline{i}a t\uline{u}a:~* et veritátem tuam n\uline{i}mis.
139. Tabéscere me fecit z\uuline{e}lus m\uline{e}us:~* quia oblíti sunt verba tua inimíci m\uline{e}i.
140. Ignítum elóquium tuum v\uuline{e}hem\uline{é}nter:~* et servus tuus diléxit \uline{i}llud.
141. Adolescéntulus sum ego \uuline{e}t cont\uline{é}mptus:~* justificatiónes tuas non sum obl\uline{í}tus.
142. Justítia tua, justítia \uuline{i}n æt\uline{é}rnum:~* et lex tua v\uline{é}ritas.
143. Tribulátio, et angústia \uuline{i}nven\uline{é}runt me:~* mandáta tua meditátio m\uline{e}a est.
144. Æquitas testimónia tua \uuline{i}n æt\uline{é}rnum:~* intelléctum da mihi, et v\uline{i}vam.
145. Clamávi in toto corde meo, exáud\uuline{i} me, D\uline{ó}mine:~* justificatiónes tuas re\uline{quí}ram.
146. Clamávi ad te, s\uuline{a}lvum m\uline{e} fac:~* ut custódiam mandáta t\uline{u}a.
147. Prævéni in maturitáte, \uuline{e}t clam\uline{á}vi:~* quia in verba tua supersper\uline{á}vi.
148. Prævenérunt óculi mei ad t\uuline{e} dil\uline{ú}culo:~* ut meditárer elóquia t\uline{u}a.
149. Vocem meam audi secúndum misericórdiam t\uuline{u}am, D\uline{ó}mine:~* et secúndum judícium tuum vivífic\uline{a} me.
150. Appropinquavérunt persequéntes me in\uuline{i}quit\uline{á}ti:~* a lege autem tua longe f\uline{a}cti sunt.
151. Prope \uuline{e}s tu, D\uline{ó}mine:~* et omnes viæ tuæ v\uline{é}ritas.
152. Inítio cognóvi de testimón\uuline{i}is t\uline{u}is:~* quia in ætérnum fundásti \uline{e}a.
153. Vide humilitátem meam, et \uuline{é}rip\uline{e} me:~* quia legem tuam non sum obl\uline{í}tus.
154. Júdica judícium meum, et r\uuline{é}dim\uline{e} me:~* propter elóquium tuum vivífic\uline{a} me.
155. Longe a peccatór\uuline{i}bus s\uline{a}lus:~* quia justificatiónes tuas non exquisi\uline{é}runt.
156. Misericórdiæ tuæ m\uuline{u}ltæ, D\uline{ó}mine:~* secúndum judícium tuum vivífic\uline{a} me.
157. Multi qui persequúntur me, et tr\uuline{í}bul\uline{a}nt me:~* a testimóniis tuis non declin\uline{á}vi.
158. Vidi prævaricántes, et t\uuline{a}besc\uline{é}bam:~* quia elóquia tua non custodi\uline{é}runt.
159. Vide quóniam mandáta tua dil\uuline{é}xi, D\uline{ó}mine:~* in misericórdia tua vivífic\uline{a} me.
160. Princípium verbórum tu\uuline{ó}rum, v\uline{é}ritas:~* in ætérnum ómnia judícia justítiæ t\uline{u}æ.
161. Príncipes persecúti s\uuline{u}nt me gr\uline{a}tis:~* et a verbis tuis formidávit cor m\uline{e}um.
162. Lætábor ego super eló\uuline{qui}a t\uline{u}a:~* sicut qui invénit spólia m\uline{u}lta.
163. Iniquitátem ódio hábui, et ab\uuline{o}min\uline{á}tus sum:~* legem autem tuam dil\uline{é}xi.
164. Sépties in die laudem d\uuline{i}xi t\uline{i}bi,~* super judícia justítiæ t\uline{u}æ.
165. Pax multa diligéntibus l\uuline{e}gem t\uline{u}am:~* et non est illis sc\uline{á}ndalum.
166. Exspectábam salutáre t\uuline{u}um, D\uline{ó}mine:~* et mandáta tua dil\uline{é}xi.
167. Custodívit ánima mea testimón\uuline{i}a t\uline{u}a:~* et diléxit ea vehem\uline{é}nter.
168. Servávi mandáta tua, et testimón\uuline{i}a t\uline{u}a:~* quia omnes viæ meæ in conspéctu t\uline{u}o.
169. Appropínquet deprecátio mea in conspéctu t\uuline{u}o, D\uline{ó}mine:~* juxta elóquium tuum da mihi intell\uline{é}ctum.
170. Intret postulátio mea in consp\uuline{é}ctu t\uline{u}o:~* secúndum elóquium tuum érip\uline{e} me.
171. Eructábunt lábia m\uuline{e}a h\uline{y}mnum,~* cum docúeris me justificatiónes t\uline{u}as.
172. Pronuntiábit lingua mea eló\uuline{qui}um t\uline{u}um:~* quia ómnia mandáta tua \uline{ǽ}quitas.
173. Fiat manus tu\uuline{a} ut s\uline{a}lvet me:~* quóniam mandáta tua el\uline{é}gi.
174. Concupívi salutáre t\uuline{u}um, D\uline{ó}mine:~* et lex tua meditátio m\uline{e}a est.
175. Vivet ánima mea, \uuline{e}t laud\uline{á}bit te:~* et judícia tua adjuv\uline{á}bunt me.
176. Errávi, sicut ov\uuline{i}s, quæ p\uline{é}riit:~* quæ re servum tuum, quia mandáta tua non sum obl\uline{í}tus.
177. Glória Patr\uuline{i}, et F\uline{í}lio,~* et Spirítui S\uline{a}ncto.
178. Sicut erat in princípio, et n\uuline{u}nc, et s\uline{e}mper,~* et in sǽcula sæculórum. \uline{A}men.
