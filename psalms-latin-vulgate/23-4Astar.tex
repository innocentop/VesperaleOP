2. Quia ipse super mária fund\uuline{á}vit \uline{e}um:~* et super flúmina præp\uuline{a}rávit \uline{e}um.
3. Quis ascéndet in m\uuline{o}ntem D\uline{ó}mini?~* aut quis stabit in loc\uuline{o} sancto \uline{e}jus?
4. Innocens mánibus et m\uuline{u}ndo c\uline{o}rde,~* qui non accépit in vano ánimam suam, nec jurávit in dolo pr\uuline{ó}ximo s\uline{u}o.
5. Hic accípiet benedictión\uuline{e}m a D\uline{ó}mino:~* et misericórdiam a Deo, sal\uuline{u}tári s\uline{u}o.
6. Hæc est generátio quærént\uuline{i}um \uline{e}um,~* quæréntium fáci\uuline{e}m Dei J\uline{a}cob.
7. Attóllite portas, príncipes, vestras,~† et elevámini, portæ \uuline{æ}tern\uline{á}les:~* et intro\uuline{í}bit Rex gl\uline{ó}riæ.
8. Quis est ist\uuline{e} Rex gl\uline{ó}riæ?~* Dóminus fortis et potens: Dóminus p\uuline{o}tens in pr\uline{ǽ}lio.
9. Attóllite portas, príncipes, vestras,~† et elevámini, portæ \uuline{æ}tern\uline{á}les:~* et intro\uuline{í}bit Rex gl\uline{ó}riæ.
10. Quis est ist\uuline{e} Rex gl\uline{ó}riæ?~* Dóminus virtútum ips\uuline{e} est Rex gl\uline{ó}riæ.
11. Glória Patr\uuline{i}, et F\uline{í}lio,~* et Spir\uuline{í}tui S\uline{a}ncto.
12. Sicut erat in princípio, et n\uuline{u}nc, et s\uline{e}mper,~* et in sǽcula sæc\uuline{u}lórum. \uline{A}men.
