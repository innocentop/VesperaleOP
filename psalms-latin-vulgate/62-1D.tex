2. Sitívit in te \uline{á}nima m\uline{e}a,~* quam multiplíciter tibi c\uuline{a}ro m\uline{e}a.\par 
3. In terra desérta, et ínvia, et ina\uline{quó}sa:~† sic in sancto app\uline{á}rui t\uline{i}bi,~* ut vidérem virtútem tuam, et glór\uuline{i}am t\uline{u}am.\par 
4. Quóniam mélior est misericórdia tua s\uline{u}per v\uline{i}tas:~* lábia me\uuline{a} laud\uline{á}bunt te.\par 
5. Sic benedícam te in v\uline{i}ta m\uline{e}a:~* et in nómine tuo levábo m\uuline{a}nus m\uline{e}as.\par 
6. Sicut ádipe et pinguédine repleátur \uline{á}nima m\uline{e}a:~* et lábiis exsultatiónis laudáb\uuline{i}t os m\uline{e}um.\par 
7. Si memor fui tui super stratum \uline{me}um,~† in matutínis medit\uline{á}bor \uline{i}n te:~* quia fuísti adj\uuline{ú}tor m\uline{e}us.\par 
8. Et in velaménto alárum tuárum exsul\uline{tá}bo,~† adhǽsit ánima m\uline{e}a p\uline{o}st te:~* me suscépit déxt\uuline{e}ra t\uline{u}a.\par 
9. Ipsi vero in vanum quæsiérunt ánimam \uline{me}am,~† introíbunt in inferi\uline{ó}ra t\uline{e}rræ:~* tradéntur in manus gládii, partes vúlp\uuline{i}um \uline{e}runt.\par 
10. Rex vero lætábitur in \uline{De}o,~† laudabúntur omnes qui j\uline{u}rant in \uline{e}o:~* quia obstrúctum est os loquénti\uuline{u}m in\uline{í}qua.\par 
11. Glória P\uline{a}tri, et F\uline{í}lio,~* et Spirít\uuline{u}i S\uline{a}ncto.\par 
12. Sicut erat in princípio, et n\uline{u}nc, et s\uline{e}mper,~* et in sǽcula sæcul\uuline{ó}rum. \uline{A}men.\par 
