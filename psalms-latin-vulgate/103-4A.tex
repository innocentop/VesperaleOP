2. Confessiónem, et decórem \uuline{i}ndu\uline{í}sti:~* amíctus lúmine sic\uuline{u}t vestim\uline{é}nto.\par 
3. Exténdens cælum s\uuline{i}cut p\uline{e}llem:~* qui tegis aquis super\uuline{i}óra \uline{e}jus.\par 
4. Qui ponis nubem asc\uuline{é}nsum t\uline{u}um:~* qui ámbulas super p\uuline{e}nnas vent\uline{ó}rum.\par 
5. Qui facis ángelos t\uuline{u}os, sp\uline{í}ritus:~* et minístros tuos \uuline{i}gnem ur\uline{é}ntem.\par 
6. Qui fundásti terram super stabilit\uuline{á}tem s\uline{u}am:~* non inclinábitur in s\uuline{ǽ}culum s\uline{ǽ}culi.\par 
7. Abýssus, sicut vestiméntum, am\uuline{í}ctus \uline{e}jus:~* super mont\uuline{e}s stabunt \uline{a}quæ.\par 
8. Ab increpatióne t\uuline{u}a f\uline{ú}gient:~* a voce tonítrui tu\uuline{i} formid\uline{á}bunt.\par 
9. Ascéndunt montes: et desc\uuline{é}ndunt c\uline{a}mpi~* in locum, quem f\uuline{u}ndásti \uline{e}is.\par 
10. Términum posuísti, quem non transgr\uuline{e}di\uline{é}ntur:~* neque converténtur op\uuline{e}ríre t\uline{e}rram.\par 
11. Qui emíttis fontes \uuline{i}n conv\uline{á}llibus:~* inter médium móntium pertr\uuline{a}nsíbunt \uline{a}quæ.\par 
12. Potábunt omnes bést\uuline{i}æ \uline{a}gri:~* exspectábunt ónagri \uuline{i}n siti s\uline{u}a.\par 
13. Super ea vólucres cæli h\uuline{a}bit\uline{á}bunt:~* de médio petrár\uuline{u}m dabunt v\uline{o}ces.\par 
14. Rigans montes de superiór\uuline{i}bus s\uline{u}is:~* de fructu óperum tuórum sati\uuline{á}bitur t\uline{e}rra:\par 
15. Prodúcens fœn\uuline{u}m jum\uline{é}ntis:~* et herbam serv\uuline{i}túti h\uline{ó}minum:\par 
16. Ut edúcas pan\uuline{e}m de t\uline{e}rra:~* et vinum lætíf\uuline{i}cet cor h\uline{ó}minis:\par 
17. Ut exhílaret fáci\uuline{e}m in \uline{ó}leo:~* et panis cor hóm\uuline{i}nis conf\uline{í}rmet.\par 
18. Saturabúntur ligna campi, et cedri Líbani, \uuline{qua}s plant\uline{á}vit:~* illic pásseres n\uuline{i}dific\uline{á}bunt.\par 
19. Heródii domus dux \uuline{e}st e\uline{ó}rum:~* montes excélsi cervis: petra refúgi\uuline{u}m herin\uline{á}ciis.\par 
20. Fecit lun\uuline{a}m in t\uline{é}mpora:~* sol cognóvit \uuline{o}ccásum s\uline{u}um.\par 
21. Posuísti ténebras, et f\uuline{a}cta \uline{e}st nox:~* in ipsa pertransíbunt omnes b\uuline{é}stiæ s\uline{i}lvæ.\par 
22. Cátuli leónum rugiént\uuline{e}s, ut r\uline{á}piant:~* et quærant a De\uuline{o} escam s\uline{i}bi.\par 
23. Ortus est sol, et c\uuline{o}ngreg\uline{á}ti sunt:~* et in cubílibus suis c\uuline{o}llocab\uline{ú}ntur.\par 
24. Exíbit homo ad \uuline{o}pus s\uline{u}um:~* et ad operatiónem suam \uuline{u}sque ad v\uline{é}sperum.\par 
25. Quam magnificáta sunt ópera t\uuline{u}a, D\uline{ó}mine!~* ómnia in sapiéntia fecísti: impléta est terra possess\uuline{i}óne t\uline{u}a.\par 
26. Hoc mare magnum, et spati\uuline{ó}sum m\uline{á}nibus:~* illic reptília, quor\uuline{u}m non est n\uline{ú}merus.\par 
27. Animália pusíll\uuline{a} cum m\uline{a}gnis:~* illic nav\uuline{e}s pertrans\uline{í}bunt.\par 
28. Draco iste, quem formásti ad illud\uuline{é}ndum \uline{e}i:~* ómnia a te exspéctant ut des illis \uuline{e}scam in t\uline{é}mpore.\par 
29. Dante te \uuline{i}llis, c\uline{ó}lligent:~* aperiénte te manum tuam, ómnia implebúnt\uuline{u}r bonit\uline{á}te.\par 
30. Averténte autem te fáciem, t\uuline{u}rbab\uline{ú}ntur:~* áuferes spíritum eórum, et defícient, et in púlverem su\uuline{u}m revert\uline{é}ntur.\par 
31. Emíttes spíritum tuum, et cr\uuline{e}ab\uline{ú}ntur:~* et renovábis f\uuline{á}ciem t\uline{e}rræ.\par 
32. Sit glória Dómin\uuline{i} in s\uline{ǽ}culum:~* lætábitur Dóminus in op\uuline{é}ribus s\uline{u}is:\par 
33. Qui réspicit terram, et facit \uuline{e}am tr\uline{é}mere:~* qui tangit m\uuline{o}ntes, et f\uline{ú}migant.\par 
34. Cantábo Dómino in v\uuline{i}ta m\uline{e}a:~* psallam Deo me\uuline{o}, quámdi\uline{u} sum.\par 
35. Jucúndum sit ei eló\uuline{qui}um m\uline{e}um:~* ego vero delect\uuline{á}bor in D\uline{ó}mino.\par 
36. Defíciant peccatóres a terra, et iníqui it\uuline{a} ut n\uline{o}n sint:~* bénedic, ánim\uuline{a} mea, D\uline{ó}mino.\par 
37. Glória Patr\uuline{i}, et F\uline{í}lio,~* et Spir\uuline{í}tui S\uline{a}ncto.\par 
38. Sicut erat in princípio, et n\uuline{u}nc, et s\uline{e}mper,~* et in sǽcula sæc\uuline{u}lórum. \uline{A}men.\par 
