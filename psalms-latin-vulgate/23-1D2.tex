2. Quia ipse super mária fund\uline{á}vit \uline{e}um:~* et super flúmina præpar\uuline{á}vit \uline{e}um.
3. Quis ascéndet in m\uline{o}ntem D\uline{ó}mini?~* aut quis stabit in loco s\uuline{a}ncto \uline{e}jus?
4. Innocens mánibus et m\uline{u}ndo c\uline{o}rde,~* qui non accépit in vano ánimam suam, nec jurávit in dolo próx\uuline{i}mo s\uline{u}o.
5. Hic accípiet benedicti\uline{ó}nem a D\uline{ó}mino:~* et misericórdiam a Deo, salut\uuline{á}ri s\uline{u}o.
6. Hæc est generátio quær\uline{é}ntium \uline{e}um,~* quæréntium fáciem D\uuline{e}i J\uline{a}cob.
7. Attóllite portas, príncipes, vestras,~† et elevámini, portæ \uline{æ}tern\uline{á}les:~* et introíbit R\uuline{e}x glór\uline{i}æ.
8. Quis est \uline{i}ste Rex gl\uline{ó}riæ?~* Dóminus fortis et potens: Dóminus potens \uuline{i}n prǽl\uline{i}o.
9. Attóllite portas, príncipes, vestras,~† et elevámini, portæ \uline{æ}tern\uline{á}les:~* et introíbit R\uuline{e}x glór\uline{i}æ.
10. Quis est \uline{i}ste Rex gl\uline{ó}riæ?~* Dóminus virtútum ipse est R\uuline{e}x glór\uline{i}æ.
11. Glória P\uline{a}tri, et F\uline{í}lio,~* et Spirít\uuline{u}i S\uline{a}ncto.
12. Sicut erat in princípio, et n\uline{u}nc, et s\uline{e}mper,~* et in sǽcula sæcul\uuline{ó}rum. \uline{A}men.
