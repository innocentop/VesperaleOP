2. Inténde voci oratiónis m\uline{e}æ:~* Rex meus et D\uline{e}us m\uline{e}us.
3. Quóniam ad te or\uline{á}bo:~* Dómine, mane exáudies v\uline{o}cem m\uline{e}am.
4. Mane astábo tibi et vid\uline{é}bo:~* quóniam non Deus volens iniquit\uline{á}tem t\uline{u} es.
5. Neque habitábit juxta te mal\uline{í}gnus:~* neque permanébunt injústi ante \uline{ó}culos t\uline{u}os.
6. Odísti omnes, qui operántur iniquit\uline{á}tem:~* perdes omnes, qui lo\uline{quú}ntur mend\uline{á}cium.
7. Virum sánguinum et dolósum abominábitur D\uline{ó}minus:~* ego autem in multitúdine miseric\uline{ó}rdiæ t\uline{u}æ.
8. Introíbo in domum t\uline{u}am:~* adorábo ad templum sanctum tuum in tim\uline{ó}re t\uline{u}o.
9. Dómine, deduc me in justítia t\uline{u}a:~* propter inimícos meos dírige in conspéctu tuo v\uline{i}am m\uline{e}am.
10. Quóniam non est in ore eórum v\uline{é}ritas:~* cor e\uline{ó}rum v\uline{a}num est.
11. Sepúlcrum patens est guttur eórum,~† linguis suis dolóse ag\uline{é}bant,~* júdica \uline{i}llos, D\uline{e}us.
12. Décidant a cogitatiónibus suis,~† secúndum multitúdinem impietátum eórum expélle \uline{e}os,~* quóniam irritav\uline{é}runt te, D\uline{ó}mine.
13. Et læténtur omnes, qui sperant \uline{i}n te,~* in ætérnum exsultábunt: et habit\uline{á}bis in \uline{e}is.
14. Et gloriabúntur in te omnes, qui díligunt nomen t\uline{u}um:~* quóniam tu bened\uline{í}ces j\uline{u}sto.
15. Dómine, ut scuto bonæ voluntátis t\uline{u}æ~* c\uline{o}ron\uline{á}sti nos.
16. Glória Patri, et F\uline{í}lio,~* et Spir\uline{í}tui S\uline{a}ncto.
17. Sicut erat in princípio, et nunc, et s\uline{e}mper,~* et in sǽcula sæcul\uline{ó}rum. \uline{A}men.
