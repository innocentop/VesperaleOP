2. Sicut déficit fumus, def\uline{í}ciant:~* sicut fluit cera a fácie ignis, sic péreant peccatóres a fáci\uuline{e} D\uline{e}i.
3. Et justi epuléntur, et exsúltent in conspéctu D\uline{e}i:~* et delecténtur in læt\uuline{í}t\uline{i}a.
4. Cantáte Deo, psalmum dícite nómini \uline{e}jus:~* iter fácite ei, qui ascéndit super occásum: Dóminus nom\uuline{e}n \uline{i}lli.
5. Exsultáte in conspéctu \uline{e}jus:~* turbabúntur a fácie ejus, patris orphanórum et júdicis vid\uuline{u}\uline{á}rum.
6. Deus in loco sancto s\uline{u}o:~* Deus, qui inhabitáre facit uníus moris \uuline{i}n d\uline{o}mo:
7. Qui edúcit vinctos in fortit\uline{ú}dine,~* simíliter eos qui exásperant, qui hábitant in s\uuline{e}p\uline{ú}lcris.
8. Deus, cum egrederéris in conspéctu pópuli t\uline{u}i,~* cum pertransíres in d\uuline{e}s\uline{é}rto:
9. Terra mota est, étenim cæli distillavérunt a fácie Dei S\uline{í}nai,~* a fácie Dei \uuline{I}sr\uline{a}ël.
10. Plúviam voluntáriam segregábis, Deus, hereditáti t\uline{u}æ:~* et infirmáta est, tu vero perfecíst\uuline{i} \uline{e}am.
11. Animália tua habitábunt in \uline{e}a:~* parásti in dulcédine tua páuper\uuline{i}, D\uline{e}us.
12. Dóminus dabit verbum evangeliz\uline{á}ntibus,~* virtút\uuline{e} m\uline{u}lta.
13. Rex virtútum dilécti dil\uline{é}cti:~* et speciéi domus divídere sp\uuline{ó}l\uline{i}a.
14. Si dormiátis inter médios cleros, pennæ colúmbæ deargent\uline{á}tæ,~* et posterióra dorsi ejus in pallór\uuline{e} \uline{au}ri.
15. Dum discérnit cæléstis reges super eam, nive dealbabúntur in S\uline{e}lmon:~* mons Dei, m\uuline{o}ns p\uline{i}nguis.
16. Mons coagulátus, mons p\uline{i}nguis:~* ut quid suspicámini montes coag\uuline{u}l\uline{á}tos?
17. Mons, in quo beneplácitum est Deo habitáre in \uline{e}o:~* étenim Dóminus habitábit \uuline{i}n f\uline{i}nem.
18. Currus Dei decem míllibus múltiplex, míllia læt\uline{á}ntium:~* Dóminus in eis in Sina \uuline{i}n s\uline{a}ncto.
19. Ascendísti in altum, cepísti captivit\uline{á}tem:~* accepísti dona in hom\uuline{í}n\uline{i}bus.
20. Etenim non cred\uline{é}ntes,~* inhabitáre Dómin\uuline{u}m D\uline{e}um.
21. Benedíctus Dóminus die quot\uline{í}die:~* prósperum iter fáciet nobis Deus salutárium n\uuline{o}str\uline{ó}rum.
22. Deus noster, Deus salvos faci\uline{é}ndi:~* et Dómini Dómini éxit\uuline{u}s m\uline{o}rtis.
23. Verúmtamen Deus confrínget cápita inimicórum su\uline{ó}rum:~* vérticem capílli perambulántium in delíct\uuline{i}s s\uline{u}is.
24. Dixit Dóminus: Ex Basan conv\uline{é}rtam,~* convértam in profúnd\uuline{u}m m\uline{a}ris:
25. Ut intingátur pes tuus in s\uline{á}nguine:~* lingua canum tuórum ex inimícis, \uuline{a}b \uline{i}pso.
26. Vidérunt ingréssus tuos, D\uline{e}us:~* ingréssus Dei mei: regis mei qui est \uuline{i}n s\uline{a}ncto.
27. Prævenérunt príncipes conjúncti psall\uline{é}ntibus:~* in médio juvenculárum tympanistr\uuline{i}\uline{á}rum.
28. In ecclésiis benedícite Deo D\uline{ó}mino,~* de fóntibus \uuline{I}sr\uline{a}ël.
29. Ibi Bénjamin adolesc\uline{é}ntulus:~* in mentis \uuline{e}xc\uline{é}ssu.
30. Príncipes Juda, duces e\uline{ó}rum:~* príncipes Zábulon, príncipes N\uuline{é}pht\uline{a}li.
31. Manda, Deus, virtúti t\uline{u}æ:~* confírma hoc, Deus, quod operátus es \uuline{i}n n\uline{o}bis.
32. A templo tuo in Jer\uline{ú}salem,~* tibi ófferent reges m\uuline{ú}n\uline{e}ra.
33. Increpa feras arúndinis, congregátio taurórum in vaccis popul\uline{ó}rum:~* ut exclúdant eos, qui probáti sunt \uuline{a}rg\uline{é}nto.
34. Díssipa Gentes, quæ bella volunt: vénient legáti ex Æg\uline{ý}pto:~* Æthiópia prævéniet manus ej\uuline{u}s D\uline{e}o.
35. Regna terræ, cantáte D\uline{e}o:~* psállite D\uuline{ó}m\uline{i}no.
36. Psállite Deo, qui ascéndit super cælum c\uline{æ}li,~* ad Or\uuline{i}\uline{é}ntem.
37. Ecce dabit voci suæ vocem virtútis, date glóriam Deo super \uline{I}sraël,~* magnificéntia ejus, et virtus ejus in n\uuline{ú}b\uline{i}bus.
38. Mirábilis Deus in sanctis suis, Deus Israël ipse dabit virtútem, et fortitúdinem plebi s\uline{u}æ,~* benedíct\uuline{u}s D\uline{e}us.
39. Glória Patri, et F\uline{í}lio,~* et Spirítu\uuline{i} S\uline{a}ncto.
40. Sicut erat in princípio, et nunc, et s\uline{e}mper,~* et in sǽcula sæculór\uuline{u}m. \uline{A}men.
