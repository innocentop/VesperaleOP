2. Apprehénde arm\uuline{a} et sc\uline{u}tum:~* et exsúrge in adjut\uuline{ó}rium m\uline{i}hi.
3. Effúnde frámeam, et conclúde advérsus eos, qui p\uuline{e}rse\uline{quú}ntur me:~* dic ánimæ meæ: Sal\uuline{u}s tua \uline{e}go sum.
4. Confundántur et rev\uuline{e}re\uline{á}ntur,~* quæréntes \uuline{á}nimam m\uline{e}am.
5. Avertántur retrórsum, et c\uuline{o}nfund\uline{á}ntur~* cogitánt\uuline{e}s mihi m\uline{a}la.
6. Fiant tamquam pulvis ante fác\uuline{i}em v\uline{e}nti:~* et Angelus Dómini c\uuline{o}árctans \uline{e}os.
7. Fiat via illórum ténebr\uuline{æ} et l\uline{ú}bricum:~* et Angelus Dómini p\uuline{é}rsequens \uline{e}os.
8. Quóniam gratis abscondérunt mihi intéritum lá\uuline{que}i s\uline{u}i:~* supervácue exprobravérunt \uuline{á}nimam m\uline{e}am.
9. Véniat illi láqueus, quem ignórat:~† et cáptio, quam abscóndit, appreh\uuline{é}ndat \uline{e}um:~* et in láqueum c\uuline{a}dat in \uline{i}psum.
10. Anima autem mea exsultáb\uuline{i}t in D\uline{ó}mino:~* et delectábitur super sal\uuline{u}tári s\uline{u}o.
11. Omnia ossa m\uuline{e}a d\uline{i}cent:~* Dómine, quis s\uuline{í}milis t\uline{i}bi?
12. Erípiens ínopem de manu forti\uuline{ó}rum \uline{e}jus:~* egénum et páuperem a diripi\uuline{é}ntibus \uline{e}um.
13. Surgéntes test\uuline{e}s in\uline{í}qui,~* quæ ignorábam \uuline{i}nterrog\uline{á}bant me.
14. Retribuébant mihi mal\uuline{a} pro b\uline{o}nis:~* sterilitátem \uuline{á}nimæ m\uline{e}æ.
15. Ego autem cum mihi mol\uuline{é}sti \uline{e}ssent,~* indu\uuline{é}bar cil\uline{í}cio.
16. Humiliábam in jejúnio án\uuline{i}mam m\uline{e}am:~* et orátio mea in sinu me\uuline{o} convert\uline{é}tur.
17. Quasi próximum, et quasi fratrem nostrum, sic c\uuline{o}mplac\uline{é}bam:~* quasi lugens et contristátus, sic h\uuline{u}mili\uline{á}bar.
18. Et advérsum me lætáti sunt, et c\uuline{o}nven\uline{é}runt:~* congregáta sunt super me flagélla, \uuline{e}t ignor\uline{á}vi.
19. Dissipáti sunt, nec compúncti,~† tentavérunt me, subsannavérunt me subsann\uuline{a}ti\uline{ó}ne:~* frenduérunt super me d\uuline{é}ntibus s\uline{u}is.
20. Dómine, quand\uuline{o} resp\uline{í}cies?~* restítue ánimam meam a malignitáte eórum, a leónibus \uuline{ú}nicam m\uline{e}am.
21. Confitébor tibi in ecclés\uuline{i}a m\uline{a}gna,~* in pópulo gr\uuline{a}vi laud\uline{á}bo te.
22. Non supergáudeant mihi qui adversántur mih\uuline{i} in\uline{í}que:~* qui odérunt me gratis et \uuline{á}nnuunt \uline{ó}culis.
23. Quóniam mihi quidem pacífice l\uuline{o}queb\uline{á}ntur:~* et in iracúndia terræ loquéntes, dol\uuline{o}s cogit\uline{á}bant.
24. Et dilatavérunt super m\uuline{e} os s\uline{u}um:~* dixérunt: Euge, euge, vidérunt \uuline{ó}culi n\uline{o}stri.
25. Vidísti, Dómin\uuline{e}, ne s\uline{í}leas:~* Dómine, ne d\uuline{i}scédas \uline{a} me.
26. Exsúrge et inténde judíc\uuline{i}o m\uline{e}o:~* Deus meus, et Dóminus meus \uuline{i}n causam m\uline{e}am.
27. Júdica me secúndum justítiam tuam, Dómine, D\uuline{e}us m\uline{e}us,~* et non superg\uuline{áu}deant m\uline{i}hi.
28. Non dicant in córdibus suis:~† Euge, euge, án\uuline{i}mæ n\uline{o}stræ:~* nec dicant: Devor\uuline{á}vimus \uline{e}um.
29. Erubéscant et revere\uuline{á}ntur s\uline{i}mul,~* qui gratulánt\uuline{u}r malis m\uline{e}is.
30. Induántur confusióne et r\uuline{e}ver\uline{é}ntia~* qui magna l\uuline{o}quúntur s\uline{u}per me.
31. Exsúltent et læténtur qui volunt justít\uuline{i}am m\uline{e}am:~* et dicant semper: Magnificétur Dóminus qui volunt pac\uuline{e}m servi \uline{e}jus.
32. Et lingua mea meditábitur justít\uuline{i}am t\uline{u}am,~* tota di\uuline{e} laudem t\uline{u}am.
33. Glória Patr\uuline{i}, et F\uline{í}lio,~* et Spir\uuline{í}tui S\uline{a}ncto.
34. Sicut erat in princípio, et n\uuline{u}nc, et s\uline{e}mper,~* et in sǽcula sæc\uuline{u}lórum. \uline{A}men.
