2. Confessiónem, et decórem \uuline{i}ndu\uline{í}sti:~* amíctus lúmine sicut vestim\uline{é}nto.
3. Exténdens cælum s\uuline{i}cut p\uline{e}llem:~* qui tegis aquis superióra \uline{e}jus.
4. Qui ponis nubem asc\uuline{é}nsum t\uline{u}um:~* qui ámbulas super pennas vent\uline{ó}rum.
5. Qui facis ángelos t\uuline{u}os, sp\uline{í}ritus:~* et minístros tuos ignem ur\uline{é}ntem.
6. Qui fundásti terram super stabilit\uuline{á}tem s\uline{u}am:~* non inclinábitur in sǽculum s\uline{ǽ}culi.
7. Abýssus, sicut vestiméntum, am\uuline{í}ctus \uline{e}jus:~* super montes stabunt \uline{a}quæ.
8. Ab increpatióne t\uuline{u}a f\uline{ú}gient:~* a voce tonítrui tui formid\uline{á}bunt.
9. Ascéndunt montes: et desc\uuline{é}ndunt c\uline{a}mpi~* in locum, quem fundásti \uline{e}is.
10. Términum posuísti, quem non transgr\uuline{e}di\uline{é}ntur:~* neque converténtur operíre t\uline{e}rram.
11. Qui emíttis fontes \uuline{i}n conv\uline{á}llibus:~* inter médium móntium pertransíbunt \uline{a}quæ.
12. Potábunt omnes bést\uuline{i}æ \uline{a}gri:~* exspectábunt ónagri in siti s\uline{u}a.
13. Super ea vólucres cæli h\uuline{a}bit\uline{á}bunt:~* de médio petrárum dabunt v\uline{o}ces.
14. Rigans montes de superiór\uuline{i}bus s\uline{u}is:~* de fructu óperum tuórum satiábitur t\uline{e}rra:
15. Prodúcens fœn\uuline{u}m jum\uline{é}ntis:~* et herbam servitúti h\uline{ó}minum:
16. Ut edúcas pan\uuline{e}m de t\uline{e}rra:~* et vinum lætíficet cor h\uline{ó}minis:
17. Ut exhílaret fáci\uuline{e}m in \uline{ó}leo:~* et panis cor hóminis conf\uline{í}rmet.
18. Saturabúntur ligna campi, et cedri Líbani, \uuline{qua}s plant\uline{á}vit:~* illic pásseres nidific\uline{á}bunt.
19. Heródii domus dux \uuline{e}st e\uline{ó}rum:~* montes excélsi cervis: petra refúgium herin\uline{á}ciis.
20. Fecit lun\uuline{a}m in t\uline{é}mpora:~* sol cognóvit occásum s\uline{u}um.
21. Posuísti ténebras, et f\uuline{a}cta \uline{e}st nox:~* in ipsa pertransíbunt omnes béstiæ s\uline{i}lvæ.
22. Cátuli leónum rugiént\uuline{e}s, ut r\uline{á}piant:~* et quærant a Deo escam s\uline{i}bi.
23. Ortus est sol, et c\uuline{o}ngreg\uline{á}ti sunt:~* et in cubílibus suis collocab\uline{ú}ntur.
24. Exíbit homo ad \uuline{o}pus s\uline{u}um:~* et ad operatiónem suam usque ad v\uline{é}sperum.
25. Quam magnificáta sunt ópera t\uuline{u}a, D\uline{ó}mine!~* ómnia in sapiéntia fecísti: impléta est terra possessióne t\uline{u}a.
26. Hoc mare magnum, et spati\uuline{ó}sum m\uline{á}nibus:~* illic reptília, quorum non est n\uline{ú}merus.
27. Animália pusíll\uuline{a} cum m\uline{a}gnis:~* illic naves pertrans\uline{í}bunt.
28. Draco iste, quem formásti ad illud\uuline{é}ndum \uline{e}i:~* ómnia a te exspéctant ut des illis escam in t\uline{é}mpore.
29. Dante te \uuline{i}llis, c\uline{ó}lligent:~* aperiénte te manum tuam, ómnia implebúntur bonit\uline{á}te.
30. Averténte autem te fáciem, t\uuline{u}rbab\uline{ú}ntur:~* áuferes spíritum eórum, et defícient, et in púlverem suum revert\uline{é}ntur.
31. Emíttes spíritum tuum, et cr\uuline{e}ab\uline{ú}ntur:~* et renovábis fáciem t\uline{e}rræ.
32. Sit glória Dómin\uuline{i} in s\uline{ǽ}culum:~* lætábitur Dóminus in opéribus s\uline{u}is:
33. Qui réspicit terram, et facit \uuline{e}am tr\uline{é}mere:~* qui tangit montes, et f\uline{ú}migant.
34. Cantábo Dómino in v\uuline{i}ta m\uline{e}a:~* psallam Deo meo, quámdi\uline{u} sum.
35. Jucúndum sit ei eló\uuline{qui}um m\uline{e}um:~* ego vero delectábor in D\uline{ó}mino.
36. Defíciant peccatóres a terra, et iníqui it\uuline{a} ut n\uline{o}n sint:~* bénedic, ánima mea, D\uline{ó}mino.
37. Glória Patr\uuline{i}, et F\uline{í}lio,~* et Spirítui S\uline{a}ncto.
38. Sicut erat in princípio, et n\uuline{u}nc, et s\uline{e}mper,~* et in sǽcula sæculórum. \uline{A}men.
