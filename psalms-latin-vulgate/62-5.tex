2. Sitívit in te ánima m\uline{e}a,~* quam multiplíciter tibi c\uline{a}ro m\uline{e}a.
3. In terra desérta, et ínvia, et inaquósa:~† sic in sancto appárui t\uline{i}bi,~* ut vidérem virtútem tuam, et gl\uline{ó}riam t\uline{u}am.
4. Quóniam mélior est misericórdia tua super v\uline{i}tas:~* lábia m\uline{e}a laud\uline{á}bunt te.
5. Sic benedícam te in vita m\uline{e}a:~* et in nómine tuo levábo m\uline{a}nus m\uline{e}as.
6. Sicut ádipe et pinguédine repleátur ánima m\uline{e}a:~* et lábiis exsultatiónis laud\uline{á}bit os m\uline{e}um.
7. Si memor fui tui super stratum meum,~† in matutínis meditábor \uline{i}n te:~* quia fuísti adj\uline{ú}tor m\uline{e}us.
8. Et in velaménto alárum tuárum exsultábo,~† adhǽsit ánima mea p\uline{o}st te:~* me suscépit d\uline{é}xtera t\uline{u}a.
9. Ipsi vero in vanum quæsiérunt ánimam meam,~† introíbunt in inferióra t\uline{e}rræ:~* tradéntur in manus gládii, partes v\uline{ú}lpium \uline{e}runt.
10. Rex vero lætábitur in Deo,~† laudabúntur omnes qui jurant in \uline{e}o:~* quia obstrúctum est os loquénti\uline{u}m in\uline{í}qua.
11. Glória Patri, et F\uline{í}lio,~* et Spir\uline{í}tui S\uline{a}ncto.
12. Sicut erat in princípio, et nunc, et s\uline{e}mper,~* et in sǽcula sæcul\uline{ó}rum. \uline{A}men.
