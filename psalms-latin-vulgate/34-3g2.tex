2. Apprehénde \uline{a}rma et sc\uline{u}tum:~* et exsúrge in adjut\uline{ó}rium m\uline{i}hi.
3. Effúnde frámeam, et conclúde advérsus eos, qui perse\uline{quú}nt\uline{u}r me:~* dic ánimæ meæ: Salus t\uline{u}a \uline{e}go sum.
4. Confundántur et rev\uline{e}re\uline{á}ntur,~* quæréntes \uline{á}nimam m\uline{e}am.
5. Avertántur retrórsum, et c\uline{o}nfund\uline{á}ntur~* cogitántes m\uline{i}hi m\uline{a}la.
6. Fiant tamquam pulvis ante f\uline{á}ciem v\uline{e}nti:~* et Angelus Dómini co\uline{á}rctans \uline{e}os.
7. Fiat via illórum ténebræ et l\uline{ú}br\uline{i}cum:~* et Angelus Dómini p\uline{é}rsequens \uline{e}os.
8. Quóniam gratis abscondérunt mihi intéritum l\uline{á}quei s\uline{u}i:~* supervácue exprobravérunt \uline{á}nimam m\uline{e}am.
9. Véniat illi láqueus, quem ignórat:~† et cáptio, quam abscóndit, appreh\uline{é}ndat \uline{e}um:~* et in láqueum c\uline{a}dat in \uline{i}psum.
10. Anima autem mea exsultábit in D\uline{ó}m\uline{i}no:~* et delectábitur super salut\uline{á}ri s\uline{u}o.
11. Omnia ossa m\uline{e}a d\uline{i}cent:~* Dómine, quis s\uline{í}milis t\uline{i}bi?
12. Erípiens ínopem de manu forti\uline{ó}rum \uline{e}jus:~* egénum et páuperem a diripi\uline{é}ntibus \uline{e}um.
13. Surgéntes t\uline{e}stes in\uline{í}qui,~* quæ ignorábam int\uline{e}rrog\uline{á}bant me.
14. Retribuébant mihi m\uline{a}la pro b\uline{o}nis:~* sterilitátem \uline{á}nimæ m\uline{e}æ.
15. Ego autem cum mihi mol\uline{é}sti \uline{e}ssent,~* indu\uline{é}bar cil\uline{í}cio.
16. Humiliábam in jejúnio \uline{á}nimam m\uline{e}am:~* et orátio mea in sinu meo c\uline{o}nvert\uline{é}tur.
17. Quasi próximum, et quasi fratrem nostrum, sic c\uline{o}mplac\uline{é}bam:~* quasi lugens et contristátus, sic hum\uline{i}li\uline{á}bar.
18. Et advérsum me lætáti sunt, et c\uline{o}nven\uline{é}runt:~* congregáta sunt super me flagélla, et \uline{i}gnor\uline{á}vi.
19. Dissipáti sunt, nec compúncti,~† tentavérunt me, subsannavérunt me subsann\uline{a}ti\uline{ó}ne:~* frenduérunt super me d\uline{é}ntibus s\uline{u}is.
20. Dómine, quando resp\uline{í}c\uline{i}es?~* restítue ánimam meam a malignitáte eórum, a leónibus \uline{ú}nicam m\uline{e}am.
21. Confitébor tibi in eccl\uline{é}sia m\uline{a}gna,~* in pópulo gr\uline{a}vi laud\uline{á}bo te.
22. Non supergáudeant mihi qui adversántur m\uline{i}hi in\uline{í}que:~* qui odérunt me gratis et \uline{á}nnuunt \uline{ó}culis.
23. Quóniam mihi quidem pacífice l\uline{o}queb\uline{á}ntur:~* et in iracúndia terræ loquéntes, dolos c\uline{o}git\uline{á}bant.
24. Et dilatavérunt super m\uline{e} os s\uline{u}um:~* dixérunt: Euge, euge, vidérunt \uline{ó}culi n\uline{o}stri.
25. Vidísti, Dómine, ne s\uline{í}l\uline{e}as:~* Dómine, ne disc\uline{é}das \uline{a} me.
26. Exsúrge et inténde jud\uline{í}cio m\uline{e}o:~* Deus meus, et Dóminus meus in c\uline{au}sam m\uline{e}am.
27. Júdica me secúndum justítiam tuam, Dómine, D\uline{e}us m\uline{e}us,~* et non superg\uline{áu}deant m\uline{i}hi.
28. Non dicant in córdibus suis:~† Euge, euge, \uline{á}nimæ n\uline{o}stræ:~* nec dicant: Devor\uline{á}vimus \uline{e}um.
29. Erubéscant et revere\uline{á}ntur s\uline{i}mul,~* qui gratulántur m\uline{a}lis m\uline{e}is.
30. Induántur confusióne et rever\uline{é}nt\uline{i}a~* qui magna lo\uline{quú}ntur s\uline{u}per me.
31. Exsúltent et læténtur qui volunt just\uline{í}tiam m\uline{e}am:~* et dicant semper: Magnificétur Dóminus qui volunt pacem s\uline{e}rvi \uline{e}jus.
32. Et lingua mea meditábitur just\uline{í}tiam t\uline{u}am,~* tota die l\uline{au}dem t\uline{u}am.
33. Glória Patri, et F\uline{í}l\uline{i}o,~* et Spir\uline{í}tui S\uline{a}ncto.
34. Sicut erat in princípio, et n\uline{u}nc, et s\uline{e}mper,~* et in sǽcula sæcul\uline{ó}rum. \uline{A}men.
