2. Afférte Dómino glóriam et honórem, afférte Dómino glóriam n\uline{ó}mini \uline{e}jus:~* adoráte Dóminum in átrio s\uline{a}ncto \uline{e}jus.
3. Vox Dómini super aquas, Deus majest\uline{á}tis int\uline{ó}nuit:~* Dóminus super \uline{a}quas m\uline{u}ltas.
4. Vox Dómini \uline{i}n virt\uline{ú}te:~* vox Dómini in magn\uline{i}fic\uline{é}ntia.
5. Vox Dómini confring\uline{é}ntis c\uline{e}dros:~* et confrínget Dóminus c\uline{e}dros L\uline{í}bani.
6. Et commínuet eas tamquam v\uline{í}tulum L\uline{í}bani:~* et diléctus quemádmodum fílius \uline{u}nic\uline{ó}rnium.
7. Vox Dómini intercidéntis fl\uline{a}mmam \uline{i}gnis:~* vox Dómini concutiéntis desértum: et commovébit Dóminus des\uline{é}rtum C\uline{a}des.
8. Vox Dómini præparántis cervos, et revel\uline{á}bit cond\uline{é}nsa:~* et in templo ejus omnes d\uline{i}cent gl\uline{ó}riam.
9. Dóminus dilúvium inhabit\uline{á}re f\uline{a}cit:~* et sedébit Dóminus Rex \uline{i}n æt\uline{é}rnum.
10. Dóminus virtútem pópulo s\uline{u}o d\uline{a}bit:~* Dóminus benedícet pópulo s\uline{u}o in p\uline{a}ce.
11. Glória P\uline{a}tri, et F\uline{í}lio,~* et Spir\uline{í}tui S\uline{a}ncto.
12. Sicut erat in princípio, et n\uline{u}nc, et s\uline{e}mper,~* et in sǽcula sæcul\uline{ó}rum. \uline{A}men.
