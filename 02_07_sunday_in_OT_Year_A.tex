\documentclass[11pt,twoside]{book}

%%Page Size (rev. 08/19/2016)
%\usepackage[inner=0.5in, outer=0.5in, top=0.5in, bottom=0.5in, papersize={6in,9in}, head=12pt, headheight=30pt, headsep=5pt]{geometry}
\usepackage[inner=0.5in, outer=0.5in, top=0.5in, bottom=0.5in, papersize={5.5in,8.5in}, head=12pt, headheight=30pt, headsep=5pt]{geometry}
%% width of textblock = 324 pt / 4.5in
%% A5 = 5.8 x 8.3 inches -- if papersize is A5, then margins should be [inner=0.75in, outer=0.55in, top=0.4in, bottom=0.4in]


%%Header (rev. 4/11/2011)
\usepackage{fancyhdr}
 \pagestyle{fancy}
\renewcommand{\chaptermark}[1]{\markboth{#1}{}}
\renewcommand{\sectionmark}[1]{\markright{#1}}
 \fancyhf{}
\fancyhead[LE,RO]{\thepage}
\fancyhead[CE]{\leftmark}
\fancyhead[CO]{\rightmark}
 \fancypagestyle{plain}{ %
\fancyhf{} % remove everything
\renewcommand{\headrulewidth}{0pt} % remove lines as well
\renewcommand{\footrulewidth}{0pt}}



\usepackage[autocompile,allowdeprecated=false]{gregoriotex}
\usepackage{gregoriosyms}
\gresetgregoriofont[op]{greciliae}




%%Titles (rev. 9/4/2011) -- TOCLESS --- lets you have sections that don't appear in the table of contents

\setcounter{secnumdepth}{-1}

\usepackage[compact,nobottomtitles*]{titlesec}
\titlespacing*{\chapter}{0pt}{-30pt}{0pt}
\titlespacing*{\section}{0pt}{*0}{*1}
\titlespacing*{\subsubsection}{0pt}{*0}{*0}
\titlespacing*{\subsubsubsection}{0pt}{10pt}{*0}
\titleformat{\part} {\normalfont\Huge\sc\center}{\thechapter}{1em}{}
\titleformat{\chapter} {\normalfont\LARGE\sc\center}{\thechapter}{1em}{}
\titleformat{\section} {\normalfont\Large\sc\center}{\thesection}{1em}{}
\titleformat{\subsection} {\normalfont\Large\sc\center}{\thesubsubsection}{1em}{}
\titleformat{\subsubsection}{\normalfont\large\sc\center}{\thesubsubsubsection}{1em}{}
\titleformat{\paragraph}{\normalfont\normalsize\sc\center}{\thesubsubsubsection}{1em}{}

\newcommand{\nocontentsline}[3]{}
\newcommand{\tocless}[2]{\bgroup\let\addcontentsline=\nocontentsline#1{#2}\egroup} %% lets you have sections that don't appear in the table of contents


%%%

%%Index (rev. December 11, 2013)
\usepackage[noautomatic,nonewpage]{imakeidx}


\makeindex[name=incipit,title=Index]
\indexsetup{level=\section,toclevel=section,noclearpage}

\usepackage[indentunit=8pt,rule=.5pt,columns=2]{idxlayout}


%%Table of Contents (rev. May 16, 2011)

%\usepackage{multicol}
%\usepackage{ifthen}
%\usepackage[toc]{multitoc}

%% General settings (rev. January 19, 2015)

\usepackage{ulem}

\usepackage[latin,english]{babel}
\usepackage{lettrine}

\usepackage{paracol}

\usepackage{fontspec}

\setmainfont[Ligatures=TeX,BoldFont=MinionPro-Bold,ItalicFont=MinionPro-It, BoldItalicFont=MinionPro-BoldIt]{MinionPro-Regular-Modified.otf}



%% Style for translation line
\grechangestyle{translation}{\fontsize{10}{10}\it\selectfont}
\grechangestyle{annotation}{\fontsize{10}{10}\selectfont}
\grechangestyle{commentary}{\textnormal\selectfont}
\gresetcustosalteration{invisible}

%\grechangedim{annotationseparation}{0.1cm}{scalable}

%\GreLoadSpaceConf{smith-four}

\frenchspacing

\usepackage{indentfirst} %%%indents first line after a section

\usepackage{graphicx}
%\usepackage{tocloft}

%%Hyperref (rev. August 20, 2011)
%\usepackage[colorlinks=false,hyperindex=true,bookmarks=true]{hyperref}
\usepackage{hyperref}
\hypersetup{pdftitle={Vesperale O.P. 2016}}
\hypersetup{pdfauthor={Order of Preachers}}
\hypersetup{pdfsubject={Liturgy}}
\hypersetup{pdfkeywords={Dominican, Liturgy, Order of Preachers, Dominican Rite, Liturgia Horarum, Divine Office}}

\newlength{\drop}



\begin{document}


%%%Initial Matter within Body (20 May 2011)
\raggedbottom

%%Combination

\chapter{7\textsuperscript{th} Sunday in Ordinary Time (Year A)}
\section{Second Vespers}
    \index[Varia]{Deus in adiutorium} \label{Deus in adiutorium (Varia)} \grecommentary[0pt]{} \gresetinitiallines{1} \grechangedim{spaceabovelines}{0.25 cm}{scalable} \grechangedim{maxbaroffsettextleft}{0 cm}{scalable} \gregorioscore{chants/misc.deus_in_adjutorium-T} \grechangedim{spaceabovelines}{0 cm}{scalable} \grechangedim{maxbaroffsettextleft}{0.6 cm}{scalable}
 \subsubsection{Hymnus}  \greannotation{VIII} \index[Hymnus]{Lucis creator} \label{Lucis creator (Hymnus)} \grecommentary[10pt]{\emph{Lucis creator optime}} \gresetinitiallines{1} \grechangestyle{initial}{\fontsize{36}{36}\selectfont} \grechangedim{maxbaroffsettextleft@nobar}{12 cm}{scalable} \grechangedim{spaceabovelines}{0.5cm}{scalable} \gresetlyriccentering{syllable}   \gregorioscore{chants/hy--lucis-creator-english}    
 \subsubsection{Antiphona}  \greannotation{VII d} \index[Antiphona]{Dixit Dominus} \label{Dixit Dominus (Antiphona)} \grecommentary[0pt]{Ps 109:1} \gresetinitiallines{1} \gresetlyriccentering{vowel} \grechangedim{maxbaroffsettextleft}{0 cm}{scalable} \gregorioscore{chants/an--dixit_dominus_domino_meo--dominican-mss}  \grechangedim{maxbaroffsettextleft}{0.6 cm}{scalable}
 \subsubsection{Psalm 109} \paragraph{The Messiah, king and priest}  \index[Psalmus]{Psalm 109} \label{Psalm 109 (Psalmus)} \emph{Christ’s reign will last until all his enemies are made subject to him (1~Cor 15:25).}    \vspace{5pt} \par %%underlines for psalm tones with three movements in the second and three movements in the third sections.

\noindent The Lord’s revelation to my \uline{Mas}ter:~†~\nopagebreak

“Sit \uline{on} my right:~$\star$~\nopagebreak

your foes I will put be\uline{neath} your feet.”

\noindent The Lord will wield from \uline{Zi}on~†~\nopagebreak

your scep\uline{ter} of power:~$\star$~\nopagebreak

rule in the midst of \uline{all} your foes.

\noindent A prince from the day of your \uline{birth}~†~\nopagebreak

on the \uline{ho}ly mountains;~$\star$~\nopagebreak

from the womb before the dawn \uline{I} begot you.

\noindent The Lord has sworn an oath he will not \uline{change}.~†~\nopagebreak

“You are a \uline{priest} for ever,~$\star$~\nopagebreak

a priest like Melchize\uline{dek} of old.”

\noindent The Master standing at \uline{your} right hand~$\star$~\nopagebreak

will shatter kings in the day of \uline{his} great wrath.

\noindent He shall drink from the stream \uline{by} the wayside~$\star$~\nopagebreak

and therefore he shall lift \uline{up} his head.

\noindent Glory to the Father, and \uline{to} the Son,~$\star$~\nopagebreak

and to the \uline{Ho}ly Spirit:

\noindent as it was in the begin\uline{ning}, is now,~$\star$~\nopagebreak

and will be for ev\uline{er}. Amen.

 \subsubsection{Antiphona}  \greannotation{IV E} \index[Antiphona]{Fidelia omnia} \label{Fidelia omnia (Antiphona)} \grecommentary[0pt]{Ps 110:8} \gresetinitiallines{1} \gresetlyriccentering{vowel} \grechangedim{maxbaroffsettextleft}{0 cm}{scalable} \gregorioscore{chants/an--fidelia_omnia--dominican}  \grechangedim{maxbaroffsettextleft}{0.6 cm}{scalable}
 \subsubsection{Psalm 110} \paragraph{God’s marvelous works}  \index[Psalmus]{Psalm 110} \label{Psalm 110 (Psalmus)} \emph{We are lost in wonder at all you have done for us, our Lord and mighty God (Revelation 15:3).}    \vspace{5pt} \par \noindent I will thank the Lord with \uline{all} my heart ~$\star$~\nopagebreak

in the meeting of the just and \uline{their} assembly.

\noindent Great are the works \uline{of} the Lord, ~$\star$~\nopagebreak

to be pondered by \uline{all} who love them.



\noindent Majestic and glori\uline{ous} his work, ~$\star$~\nopagebreak

his justice stands \uline{firm} for ever.

\noindent He makes us remem\uline{ber} his wonders. ~$\star$~\nopagebreak

The Lord is compas\uline{sion} and love.



\noindent He gives food to \uline{those} who fear him; ~$\star$~\nopagebreak

keeps his covenant ev\uline{er} in mind.

\noindent He has shown his might \uline{to} his people ~$\star$~\nopagebreak

by giving them the lands \uline{of} the nations.



\noindent His works are jus\uline{tice} and truth, ~$\star$~\nopagebreak

his precepts are all \uline{of} them sure,

\noindent standing firm for ev\uline{er} and ever; ~$\star$~\nopagebreak

they are made in upright\uline{ness} and truth.



\noindent He has sent deliverance to his \uline{people}~†~\nopagebreak

and established his cove\uline{nant} for ever. ~$\star$~\nopagebreak

Holy his name, \uline{to} be feared.



\noindent To fear the Lord is the first stage of \uline{wisdom;}~†~\nopagebreak

all who do so prove \uline{them}selves wise. ~$\star$~\nopagebreak

His praise shall \uline{last} for ever!



\noindent Glory to the Father, and \uline{to} the Son,~$\star$~\nopagebreak

and to the \uline{Ho}ly Spirit:

\noindent as it was in the begin\uline{ning}, is now,~$\star$~\nopagebreak

and will be for ev\uline{er}. Amen.

 \subsubsection{Canticum} \paragraph{The wedding of the Lamb} \index[Canticum]{Salus et gloria} \label{Salus et gloria (Canticum)} \grecommentary[0pt]{Cf. Ap 19:1-2, 5-7 [AR]} \gresetinitiallines{1} \gresetlyriccentering{syllable}  \gregorioscore{chants/canticle--salus-et-honor--dom-1-et-3--english}

\newpage

 \subsubsection{Lectio brevis}     \hfill 1 Pt 1:3-5    \lettrine[lines=3]{P}{}raised be the God and Father of our Lord Jesus Christ, he who in his great mercy gave us new birth; a birth unto hope which draws its life from the resurrection of Jesus Christ from the dead; a birth to an imperishable inheritance, incapable of fading or defilement, which is kept in heaven for you who are guarded with God’s power through faith; a birth to a salvation which stands ready to be revealed in the last days.

 \subsubsection{Responsorium brevis}  \greannotation{VI} \index[Responsorium brevis]{Benedictus es, Domine} \label{Benedictus es, Domine (Responsorium brevis)} \grecommentary[0pt]{Dan 3:56 [AR]} \gresetinitiallines{1} \gresetlyriccentering{vowel}  \gregorioscore{chants/rb--benedictus_es_domine--solesmes}

 \newpage


 \subsubsection{Antiphona ad Magnificat}  \greannotation{III a} \index[Antiphona ad Magnificat]{Estote ergo} \label{Estote ergo (Antiphona ad Magnificat)} \grecommentary[0pt]{Mt 5:48 [AR]} \gresetinitiallines{1} \gresetlyriccentering{vowel}  \gregorioscore{chants/an--estote_ergo_vos--solesmes} \vspace{5pt} \emph{Be you therefore perfect, as your heavenly Father is perfect.}
 \subsubsection{Canticum Evangelicum} \paragraph{The soul rejoices in the Lord} \greannotation{III a} \index[Canticum Evangelicum]{Magnificat 8G} \label{Magnificat 3a (Canticum Evangelicum)} \grecommentary[0pt]{Lc 1:46-55} \gresetinitiallines{1} \gresetlyriccentering{vowel}  \gregorioscore{chants/magnificat8G} \vspace{10pt} 

\emph{My soul proclaims the greatness of the Lord, my spirit rejoices in God my Savior for he has looked with favor on his lowly servant. From this day all generations will call me blessed: the Almighty has done great things for me, and holy is his Name. He has mercy on those who fear him in every generation. He has shown the strength of his arm, he has scattered the proud in their conceit. He has cast down the mighty from their thrones, and has lifted up the lowly. He has filled the hungry with good things, and the rich he has sent away empty. He has come to the help of his servant Israel for he has remembered his promise of mercy, the promise he made to our fathers, to Abraham and his children for ever. Glory to the Father, and to the Son, and to the Holy Spirit: as it was in the beginning, is now, and will be for ever. Amen.}


 \subsubsection{Preces}   \index[Preces]{Week III, Sunday, Second Vespers} \label{Week III, Sunday, Second Vespers (Preces)}     \lettrine[loversize=0.15,lines=2]{T}{}he world was created by the Word of God, recreated by his redemption, and it is continually renewed by his love. Rejoicing in him we call out: Renew the wonders of your love, O Lord.
\par \Rbar. Renew the wonders of your love, O Lord.

We give thanks to God whose power is revealed in nature,
– and whose providence is revealed in history.
\par \Rbar. Renew the wonders of your love, O Lord.

Through your Son, the herald of reconciliation, the victor of the cross,
– free us from empty fear and hopelessness.
\par \Rbar. Renew the wonders of your love, O Lord.

May all those who love and pursue justice,
– work together without deceit to build a world of true peace.
\par \Rbar. Renew the wonders of your love, O Lord.

Be with the oppressed, free the captives, console the sorrowing, feed the hungry, strengthen the weak,
– in all people reveal the victory of your cross.
\par \Rbar. Renew the wonders of your love, O Lord.

After your Son’s death and burial you raised him up again in glory,
– grant that the faithful departed may live with him.
\par \Rbar. Renew the wonders of your love, O Lord.


 \newpage

 \subsubsection{Pater noster}   \index[Pater noster]{Pater noster} \label{Pater noster (Pater noster)}   \grechangedim{spaceabovelines}{0.4 cm}{scalable}  \gregorioscore{chants/or--pater_noster_a--solesmes-T} \grechangedim{spaceabovelines}{0 cm}{scalable}

\newpage

 \subsubsection{Oratio conclusiva} \lettrine[lines=3]{G}{}rant, we pray, almighty God,
that, always pondering spiritual things,
we may carry out in both word and deed
that which is pleasing to you.
Through our Lord Jesus Christ, your Son,
who lives and reigns with you in the unity of the Holy Spirit,
one God, for ever and ever. \par \Rbar. Amen.



 \subsubsection{Ritus conclusionis}         \par \Vbar. The Lord be with you. \par \Rbar. And with your spirit. \par \Vbar. May almighty God bless you, the Father, and the Son, and the Holy Spirit. \par \Rbar. Amen.
 \subsubsection{Benedicamus Domino}   \index[Benedicamus Domino]{Sundays} \label{Sundays (Benedicamus Domino)}      \gregorioscore{chants/misc.benedicamus.dominio.4-T}




 \end{document}
