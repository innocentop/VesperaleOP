\documentclass[11pt,twoside]{book}

%%Page Size (rev. 08/19/2016)
%\usepackage[inner=0.5in, outer=0.5in, top=0.5in, bottom=0.5in, papersize={6in,9in}, head=12pt, headheight=30pt, headsep=5pt]{geometry}
\usepackage[inner=0.5in, outer=0.5in, top=0.5in, bottom=0.5in, papersize={5.5in,8.5in}, head=12pt, headheight=30pt, headsep=5pt]{geometry}
%% width of textblock = 324 pt / 4.5in
%% A5 = 5.8 x 8.3 inches -- if papersize is A5, then margins should be [inner=0.75in, outer=0.55in, top=0.4in, bottom=0.4in]


%%Header (rev. 4/11/2011)
\usepackage{fancyhdr}
 \pagestyle{fancy}
\renewcommand{\chaptermark}[1]{\markboth{#1}{}}
\renewcommand{\sectionmark}[1]{\markright{\thesection\ #1}}
 \fancyhf{}
\fancyhead[LE,RO]{\thepage}
\fancyhead[CE]{Vesperale O.P.}
\fancyhead[CO]{\leftmark}
 \fancypagestyle{plain}{ %
\fancyhf{} % remove everything
\renewcommand{\headrulewidth}{0pt} % remove lines as well
\renewcommand{\footrulewidth}{0pt}}



\usepackage[autocompile,allowdeprecated=false]{gregoriotex}
\usepackage{gregoriosyms}
\gresetgregoriofont[op]{greciliae}




%%Titles (rev. 9/4/2011) -- TOCLESS --- lets you have sections that don't appear in the table of contents

\setcounter{secnumdepth}{-1}

\usepackage[compact,nobottomtitles*]{titlesec}
\titlespacing*{\chapter}{0pt}{*0}{*1}
\titlespacing*{\section}{0pt}{*0}{*1}
\titlespacing*{\subsection}{0pt}{*0}{*1}
\titlespacing*{\subsubsection}{0pt}{*0}{*1}
\titleformat{\chapter} {\normalfont\LARGE\sc\center}{\thechapter}{1em}{}
\titleformat{\section} {\normalfont\Large\sc\center}{\thesection}{1em}{}
\titleformat{\subsection} {\normalfont\large\sc\center}{\thesubsection}{1em}{}
\titleformat{\subsubsection}{\normalfont\normalsize\sc\center}{\thesubsubsection}{1em}{}

\newcommand{\nocontentsline}[3]{}
\newcommand{\tocless}[2]{\bgroup\let\addcontentsline=\nocontentsline#1{#2}\egroup} %% lets you have sections that don't appear in the table of contents


%%%

%%Index (rev. December 11, 2013)
\usepackage[noautomatic,nonewpage]{imakeidx}


\makeindex[name=incipit,title=Index]
\indexsetup{level=\section,toclevel=section,noclearpage}

\usepackage[indentunit=8pt,rule=.5pt,columns=2]{idxlayout}


%%Table of Contents (rev. May 16, 2011)

%\usepackage{multicol}
%\usepackage{ifthen}
%\usepackage[toc]{multitoc}

%% General settings (rev. January 19, 2015)

\usepackage{ulem}

\usepackage[latin,english]{babel}
\usepackage{lettrine}

\usepackage{paracol}

\usepackage{fontspec}

\setmainfont[Ligatures=TeX,BoldFont=MinionPro-Bold,ItalicFont=MinionPro-It, BoldItalicFont=MinionPro-BoldIt]{MinionPro-Regular-Modified.otf}



%% Style for translation line
\grechangestyle{translation}{\fontsize{10}{10}\it\selectfont}
\grechangestyle{annotation}{\fontsize{10}{10}\selectfont}
\grechangestyle{commentary}{\textnormal\selectfont}
\gresetcustosalteration{invisible}

%\grechangedim{annotationseparation}{0.1cm}{scalable}

%\GreLoadSpaceConf{smith-four}

\frenchspacing

\usepackage{indentfirst} %%%indents first line after a section

\usepackage{graphicx}
%\usepackage{tocloft}

%%Hyperref (rev. August 20, 2011)
%\usepackage[colorlinks=false,hyperindex=true,bookmarks=true]{hyperref}
\usepackage{hyperref}
\hypersetup{pdftitle={Vesperale O.P. 2016}}
\hypersetup{pdfauthor={Order of Preachers}}
\hypersetup{pdfsubject={Liturgy}}
\hypersetup{pdfkeywords={Dominican, Liturgy, Order of Preachers, Dominican Rite, Liturgia Horarum, Divine Office}}

\newlength{\drop}



\begin{document}


%%%Initial Matter within Body (20 May 2011)
\raggedbottom

%%Combination

\chapter{Sunday Vespers}
\section{26\textsuperscript{th} Sunday of Ordinary Time (Year C)}
\subsection{September 25, 2016} \newpage
    \index[Varia]{Deus in adiutorium} \label{Deus in adiutorium (Varia)} \grecommentary[0pt]{} \gresetinitiallines{1} \grechangedim{maxbaroffsettextleft}{0 cm}{scalable} \gregorioscore{chants/misc.deus_in_adjutorium-T} \grechangedim{maxbaroffsettextleft}{0.6 cm}{scalable}
 \subsection{Hymnus}  \greannotation{} \index[Hymnus]{Lucis creator} \label{Lucis creator (Hymnus)} \grecommentary[0pt]{} \gresetinitiallines{1} \gresetlyriccentering{syllable} \gregorioscore{chants/hy--lucis-creator-english}
 \subsection{Antiphona}  \greannotation{VII d} \index[Antiphona]{Dixit Dominus} \label{Dixit Dominus (Antiphona)} \grecommentary[0pt]{Ps 109:1} \gresetinitiallines{1} \gresetlyriccentering{vowel} \gregorioscore{chants/an--dixit_dominus_domino_meo--dominican-mss}
 \subsection{Psalm 109} \subsubsection{The Messiah, king and priest}  \index[Psalmus]{Psalm 109} \label{Psalm 109 (Psalmus)} \emph{Christ’s reign will last until all his enemies are made subject to him (1~Cor 15:25).}   \vspace{5pt} \par %%underlines for psalm tones with three movements in the second and three movements in the third sections.

\noindent The Lord’s revelation to my \uline{Mas}ter:~†~\nopagebreak

“Sit \uline{on} my right:~$\star$~\nopagebreak

your foes I will put be\uline{neath} your feet.”

\noindent The Lord will wield from \uline{Zi}on~†~\nopagebreak

your scep\uline{ter} of power:~$\star$~\nopagebreak

rule in the midst of \uline{all} your foes.

\noindent A prince from the day of your \uline{birth}~†~\nopagebreak

on the \uline{ho}ly mountains;~$\star$~\nopagebreak

from the womb before the dawn \uline{I} begot you.

\noindent The Lord has sworn an oath he will not \uline{change}.~†~\nopagebreak

“You are a \uline{priest} for ever,~$\star$~\nopagebreak

a priest like Melchize\uline{dek} of old.”

\noindent The Master standing at \uline{your} right hand~$\star$~\nopagebreak

will shatter kings in the day of \uline{his} great wrath.

\noindent He shall drink from the stream \uline{by} the wayside~$\star$~\nopagebreak

and therefore he shall lift \uline{up} his head.

\noindent Glory to the Father, and \uline{to} the Son,~$\star$~\nopagebreak

and to the \uline{Ho}ly Spirit:

\noindent as it was in the begin\uline{ning}, is now,~$\star$~\nopagebreak

and will be for ev\uline{er}. Amen.

 \subsection{Antiphona}  \greannotation{T. per.} \index[Antiphona]{Nos qui vivimus} \label{Nos qui vivimus (Antiphona)} \grecommentary[0pt]{Ps 113b:18} \gresetinitiallines{1} \gresetlyriccentering{vowel} \gregorioscore{chants/an--nos_qui_vivimus_dominican_peregrinus}
 \subsection{Psalm 113B} \subsubsection{Praise of the true God}  \index[Psalmus]{Psalm 113B} \label{Psalm 113B (Psalmus)} \emph{You have renounced idol worship to serve the living and true God (1 Thessalonians 1:9).}   \vspace{5pt} \par \noindent Not to us, Lord, \uline{not} to us,~$\star$~\nopagebreak

but to your name \uline{give} the glory

\noindent for the sake of your love \uline{and} your truth,~$\star$~\nopagebreak

lest the heathen say: “Where \uline{is} their God?”



\noindent But our God is \uline{in} the heavens;~$\star$~\nopagebreak

he does whatev\uline{er} he wills.

\noindent Their idols are sil\uline{ver} and gold,~$\star$~\nopagebreak

the work of \uline{hu}man hands.



\noindent They have mouths but they \uline{can}not speak;~$\star$~\nopagebreak

they have eyes but they \uline{can}not see;

\noindent they have ears but they \uline{can}not hear;~$\star$~\nopagebreak

they have nostrils but they \uline{can}not smell.



\noindent With their hands they cannot \uline{feel};~†~\nopagebreak

with their feet they \uline{can}not walk.~$\star$~\nopagebreak

No sound comes \uline{from} their throats.



\noindent Their makers will come to \uline{be} like them~$\star$~\nopagebreak

and so will all who \uline{trust} in them.



\noindent Sons of Israel, trust \uline{in} the Lord;~$\star$~\nopagebreak

he is their help \uline{and} their shield.

\noindent Sons of Aaron, trust \uline{in} the Lord;~$\star$~\nopagebreak

he is their help \uline{and} their shield.



\noindent You who fear him, trust \uline{in} the Lord;~$\star$~\nopagebreak

he is their help \uline{and} their shield.

\noindent He remembers us, and he will \uline{bless} us;~†~\nopagebreak

he will bless the \uline{sons} of Israel.~$\star$~\nopagebreak

He will bless the \uline{sons} of Aaron.



\noindent The Lord will bless \uline{those} who fear him,~$\star$~\nopagebreak

the little no less \uline{than} the great:

\noindent to you may the \uline{Lord} grant increase,~$\star$~\nopagebreak

to you and \uline{all} your children.



\noindent May you be blessed \uline{by} the Lord,~$\star$~\nopagebreak

the maker of hea\uline{ven} and earth.

\noindent The heavens belong \uline{to} the Lord~$\star$~\nopagebreak

but the earth he has gi\uline{ven} to men.



\noindent The dead shall not \uline{praise} the Lord,~$\star$~\nopagebreak

nor those who go down in\uline{to} the silence.

\noindent But we who live \uline{bless} the Lord~$\star$~\nopagebreak

now and for ev\uline{er}. Amen.



\noindent Glory to the Father, and \uline{to} the Son,~$\star$~\nopagebreak

and to the \uline{Ho}ly Spirit:

\noindent as it was in the begin\uline{ning}, is now,~$\star$~\nopagebreak

and will be for ev\uline{er}. Amen.

 \subsection{Canticum} \subsubsection{The wedding of the Lamb} \greannotation{VI} \index[Canticum]{Salus et gloria} \label{Salus et gloria (Canticum)} \grecommentary[0pt]{Cf. Ap 19:1-2, 5-7} \gresetinitiallines{1} \gresetlyriccentering{syllable} \gregorioscore{chants/canticle--salus-et-honor--dom-1-et-3--english}
 \subsection{Lectio brevis}     \hfill \emph{2 Th 2:13-14}   \lettrine[lines=3]{W}{}e are bound to thank God for you always, beloved brothers in the Lord, because you are the first fruits of those whom God has chosen for salvation, in holiness of spirit and fidelity to truth. He called you through our preaching of the good news so that you might achieve the glory of our Lord, Jesus Christ.

 \subsection{Responsorium brevis}  \greannotation{VI} \index[Responsorium brevis]{Benedictus es, Domine} \label{Benedictus es, Domine (Responsorium brevis)} \grecommentary[0pt]{Ps 146:5} \gresetinitiallines{1} \gresetlyriccentering{vowel} \gregorioscore{chants/rb--magnus_dominus_noster--solesmes}
\subsection{Antiphona ad Magnificat}   \index[Antiphona ad Magnificat]{Dives ille (text only)} \label{Dives ille (text only) (Antiphona ad Magnificat)} \hfill \emph{S. Gregorius}   \begin{sloppypar} \begin{paracol}{2} \sloppy \noindent \lettrine[lines=3]{D}{}ives ille guttam aquæ pétiit, qui micas panis Lázaro negávit.\switchcolumn \noindent \emph{The rich man begged for a drop of water, who had denied a morsel of bread to Lazarus.} \end{paracol} \end{sloppypar}

 \subsection{Antiphona ad Magnificat}  \greannotation{VII d} \index[Antiphona ad Magnificat]{Dives ille} \label{Dives ille (Antiphona ad Magnificat)} \grecommentary[0pt]{S. Gregorius} \gresetinitiallines{1} \gresetlyriccentering{vowel} \gregorioscore{chants/an--dives_ille--solesmes}

 \subsection{Canticum Evangelicum} \subsubsection{The soul rejoices in the Lord} \greannotation{VII d} \index[Canticum Evangelicum]{Magnificat 4E} \label{Magnificat 4E (Canticum Evangelicum)} \grecommentary[0pt]{Lc 1:46-55} \gresetinitiallines{1} \gresetlyriccentering{vowel} \gregorioscore{chants/magnificat7d} \vspace{10pt} 

\emph{My soul proclaims the greatness of the Lord, my spirit rejoices in God my Savior for he has looked with favor on his lowly servant. From this day all generations will call me blessed: the Almighty has done great things for me, and holy is his Name. He has mercy on those who fear him in every generation. He has shown the strength of his arm, he has scattered the proud in their conceit. He has cast down the mighty from their thrones, and has lifted up the lowly. He has filled the hungry with good things, and the rich he has sent away empty. He has come to the help of his servant Israel for he has remembered his promise of mercy, the promise he made to our fathers, to Abraham and his children for ever. Glory to the Father, and to the Son, and to the Holy Spirit: as it was in the beginning, is now, and will be for ever. Amen.}
 \subsection{Preces}   \index[Preces]{Week II, Sunday, Second Vespers} \label{Week II, Sunday, Second Vespers (Preces)}    \lettrine[lines=2]{C}{}hrist the Lord is our head; we are his members. In joy let us call out to him: Lord, may your kingdom come.
\par \Rbar. Lord, may your kingdom come.

Christ our Savior, make your Church a more vivid symbol of the unity of all mankind, – make it more effectively the sacrament of salvation for all peoples. \par \Rbar. Lord, may your kingdom come.

Through your presence, guide the college of bishops in union with the Pope, – give them the gifts of unity, love and peace. \par \Rbar. Lord, may your kingdom come.

Bind all Christians more closely to yourself, their divine Head, – lead them to proclaim your kingdom by the witness of their lives. \par \Rbar. Lord, may your kingdom come.

Grant peace to the world, – let every land flourish in justice and security. \par \Rbar. Lord, may your kingdom come.

Grant to the dead the glory of resurrection, – and give us a share in their happiness. \par \Rbar. Lord, may your kingdom come.
 \subsection{Pater noster}   \index[Pater noster]{Pater noster} \label{Pater noster (Pater noster)}    \gregorioscore{chants/or--pater_noster_a--solesmes-T}
 \subsection{Oratio conclusiva}   \index[Oratio conclusiva]{26th Sunday in OT} \label{26th Sunday in OT (Oratio conclusiva)}    \par \lettrine[lines=3]{O}{} God, who manifest your almighty power above all by pardoning and showing mercy, bestow, we pray, your grace abundantly upon us and make those hastening to attain your promises heirs to the treasures of heaven. Through our Lord Jesus Christ, your Son, who lives and reigns with you in the unity of the Holy Spirit, one God, for ever and ever. \par \Rbar. Amen.
 \subsection{Ritus conclusionis}        \par \Vbar. The Lord be with you. \par \Rbar. And with your spirit. \par \Vbar. May almighty God bless you, the Father, and the Son, and the Holy Spirit. \par \Rbar. Amen.
 \subsection{Benedicamus Domino}   \index[Benedicamus Domino]{Sundays} \label{Sundays (Benedicamus Domino)}    \emph{Cantor:} \par \vspace{5pt} \gregorioscore{chants/misc.benedicamus.dominio.4-T}


\end{document}
