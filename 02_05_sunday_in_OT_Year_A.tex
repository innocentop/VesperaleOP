\documentclass[11pt,twoside]{book}

%%Page Size (rev. 08/19/2016)
%\usepackage[inner=0.5in, outer=0.5in, top=0.5in, bottom=0.5in, papersize={6in,9in}, head=12pt, headheight=30pt, headsep=5pt]{geometry}
\usepackage[inner=0.5in, outer=0.5in, top=0.5in, bottom=0.5in, papersize={5.5in,8.5in}, head=12pt, headheight=30pt, headsep=5pt]{geometry}
%% width of textblock = 324 pt / 4.5in
%% A5 = 5.8 x 8.3 inches -- if papersize is A5, then margins should be [inner=0.75in, outer=0.55in, top=0.4in, bottom=0.4in]


%%Header (rev. 4/11/2011)
\usepackage{fancyhdr}
 \pagestyle{fancy}
\renewcommand{\chaptermark}[1]{\markboth{#1}{}}
\renewcommand{\sectionmark}[1]{\markright{#1}}
 \fancyhf{}
\fancyhead[LE,RO]{\thepage}
\fancyhead[CE]{\leftmark}
\fancyhead[CO]{\rightmark}
 \fancypagestyle{plain}{ %
\fancyhf{} % remove everything
\renewcommand{\headrulewidth}{0pt} % remove lines as well
\renewcommand{\footrulewidth}{0pt}}


\usepackage[autocompile,allowdeprecated=false]{gregoriotex}
\usepackage{gregoriosyms}
\gresetgregoriofont[op]{greciliae}




%%Titles (rev. 9/4/2011) -- TOCLESS --- lets you have sections that don't appear in the table of contents

\setcounter{secnumdepth}{-1}

\usepackage[compact,nobottomtitles*]{titlesec}
\titlespacing*{\chapter}{0pt}{-30pt}{0pt}
\titlespacing*{\section}{0pt}{*0}{*1}
\titlespacing*{\subsection}{0pt}{*0}{*1}
\titlespacing*{\subsubsection}{0pt}{*0}{*1}
\titleformat{\chapter} {\normalfont\LARGE\sc\center}{\thechapter}{0pt}{}
\titleformat{\section} {\normalfont\Large\sc\center}{\thesection}{1em}{}
\titleformat{\subsection} {\normalfont\large\sc\center}{\thesubsection}{1em}{}
\titleformat{\subsubsection}{\normalfont\normalsize\sc\center}{\thesubsubsection}{1em}{}

\newcommand{\nocontentsline}[3]{}
\newcommand{\tocless}[2]{\bgroup\let\addcontentsline=\nocontentsline#1{#2}\egroup} %% lets you have sections that don't appear in the table of contents


%%%

%%Index (rev. December 11, 2013)
\usepackage[noautomatic,nonewpage]{imakeidx}


\makeindex[name=incipit,title=Index]
\indexsetup{level=\section,toclevel=section,noclearpage}

\usepackage[indentunit=8pt,rule=.5pt,columns=2]{idxlayout}


%%Table of Contents (rev. May 16, 2011)

%\usepackage{multicol}
%\usepackage{ifthen}
%\usepackage[toc]{multitoc}

%% General settings (rev. January 19, 2015)

\usepackage{ulem}

\usepackage[latin,english]{babel}
\usepackage{lettrine}

\usepackage{paracol}

\usepackage{fontspec}

\setmainfont[Ligatures=TeX,BoldFont=MinionPro-Bold,ItalicFont=MinionPro-It, BoldItalicFont=MinionPro-BoldIt]{MinionPro-Regular-Modified.otf}



%% Style for translation line
\grechangestyle{translation}{\fontsize{10}{10}\it\selectfont}
\grechangestyle{annotation}{\fontsize{10}{10}\selectfont}
\grechangestyle{commentary}{\textnormal\selectfont}
\gresetcustosalteration{invisible}

%\grechangedim{annotationseparation}{0.1cm}{scalable}

%\GreLoadSpaceConf{smith-four}

\frenchspacing

\usepackage{indentfirst} %%%indents first line after a section

\usepackage{graphicx}
%\usepackage{tocloft}

%%Hyperref (rev. August 20, 2011)
%\usepackage[colorlinks=false,hyperindex=true,bookmarks=true]{hyperref}
\usepackage{hyperref}
\hypersetup{pdftitle={Vesperale O.P. 2016}}
\hypersetup{pdfauthor={Order of Preachers}}
\hypersetup{pdfsubject={Liturgy}}
\hypersetup{pdfkeywords={Dominican, Liturgy, Order of Preachers, Dominican Rite, Liturgia Horarum, Divine Office}}

\newlength{\drop}



\begin{document}


%%%Initial Matter within Body (20 May 2011)
\raggedbottom

%%Combination
\chapter{5\textsuperscript{th} Sunday in Ordinary Time (Year A)}
\section{Second Vespers}
    \index[Varia]{Deus in adiutorium} \label{Deus in adiutorium (Varia)} \grecommentary[0pt]{} \gresetinitiallines{1} \grechangedim{spaceabovelines}{0.25 cm}{scalable} \grechangedim{maxbaroffsettextleft}{0 cm}{scalable} \gregorioscore{chants/misc.deus_in_adjutorium-T} \grechangedim{spaceabovelines}{0 cm}{scalable} \grechangedim{maxbaroffsettextleft}{0.6 cm}{scalable}
 \subsection{Hymnus}  \greannotation{VIII} \index[Hymnus]{Lucis creator} \label{Lucis creator (Hymnus)} \grecommentary[0pt]{} \gresetinitiallines{1} \gresetlyriccentering{syllable}  \gregorioscore{chants/hy--lucis-creator-english}
 \subsection{Antiphona}  \greannotation{VII d} \index[Antiphona]{Dixit Dominus} \label{Dixit Dominus (Antiphona)} \grecommentary[0pt]{Ps 109:1} \gresetinitiallines{1} \gresetlyriccentering{vowel} \grechangedim{maxbaroffsettextleft}{0 cm}{scalable} \gregorioscore{chants/an--dixit_dominus_domino_meo--dominican-mss}  \grechangedim{maxbaroffsettextleft}{0.6 cm}{scalable}
 \subsection{Psalm 109} \subsubsection{The Messiah, king and priest}  \index[Psalmus]{Psalm 109} \label{Psalm 109 (Psalmus)} \emph{Christ’s reign will last until all his enemies are made subject to him (1~Cor 15:25).}    \vspace{5pt} \par %%underlines for psalm tones with three movements in the second and three movements in the third sections.

\noindent The Lord’s revelation to my \uline{Mas}ter:~†~\nopagebreak

“Sit \uline{on} my right:~$\star$~\nopagebreak

your foes I will put be\uline{neath} your feet.”

\noindent The Lord will wield from \uline{Zi}on~†~\nopagebreak

your scep\uline{ter} of power:~$\star$~\nopagebreak

rule in the midst of \uline{all} your foes.

\noindent A prince from the day of your \uline{birth}~†~\nopagebreak

on the \uline{ho}ly mountains;~$\star$~\nopagebreak

from the womb before the dawn \uline{I} begot you.

\noindent The Lord has sworn an oath he will not \uline{change}.~†~\nopagebreak

“You are a \uline{priest} for ever,~$\star$~\nopagebreak

a priest like Melchize\uline{dek} of old.”

\noindent The Master standing at \uline{your} right hand~$\star$~\nopagebreak

will shatter kings in the day of \uline{his} great wrath.

\noindent He shall drink from the stream \uline{by} the wayside~$\star$~\nopagebreak

and therefore he shall lift \uline{up} his head.

\noindent Glory to the Father, and \uline{to} the Son,~$\star$~\nopagebreak

and to the \uline{Ho}ly Spirit:

\noindent as it was in the begin\uline{ning}, is now,~$\star$~\nopagebreak

and will be for ev\uline{er}. Amen.

 \subsection{Antiphona}  \greannotation{T. per.} \index[Antiphona]{Ex Ægypto} \label{Ex Ægypto (Antiphona)} \grecommentary[3pt]{Cf. Ex 13:14 [AR]} \gresetinitiallines{1} \gresetlyriccentering{vowel} \grechangedim{maxbaroffsettextleft}{0 cm}{scalable} \gregorioscore{chants/an--ex_aegypto_--solesmes--tonus-peregrinus}  \grechangedim{maxbaroffsettextleft}{0.6 cm}{scalable}
 \subsection{Psalm 113A} \subsubsection{The Israelites are delivered from the bondage of Egypt}  \index[Psalmus]{Psalm 113A} \label{Psalm 113A (Psalmus)} \emph{You too left Egypt when, at baptism, you renounced that world which is at enmity with God (Saint Augustine).}    \vspace{5pt} \par \noindent When Israel came \uline{forth} from Egypt,~$\star$~\nopagebreak

Jacob’s sons from an a\uline{li}en people,

\noindent Judah became \uline{the} Lord’s temple,~$\star$~\nopagebreak

Israel be\uline{came} his kingdom.

\noindent The sea fled \uline{at} the sight:~$\star$~\nopagebreak

the Jordan turned back \uline{on} its course,

\noindent the mountains \uline{leapt} like rams~$\star$~\nopagebreak

and the hills like \uline{year}ling sheep.

\noindent Why was it, sea, \uline{that} you fled,~$\star$~\nopagebreak

that you turned back, Jordan, \uline{on} your course?

\noindent Mountains, that you \uline{leapt} like rams,~$\star$~\nopagebreak

hills, like \uline{year}ling sheep?

\noindent Tremble, O earth, be\uline{fore} the Lord,~$\star$~\nopagebreak

in the presence of the \uline{God} of Jacob,

\noindent who turns the rock in\uline{to} a pool~$\star$~\nopagebreak

and flint into a \uline{spring} of water.

\noindent Glory to the Father, and \uline{to} the Son,~$\star$~\nopagebreak

and to the \uline{Ho}ly Spirit:

\noindent as it was in the begin\uline{ning}, is now,~$\star$~\nopagebreak

and will be for ev\uline{er}. Amen.
 \subsection{Canticum} \subsubsection{The wedding of the Lamb} \greannotation{VI} \index[Canticum]{Salus et gloria} \label{Salus et gloria (Canticum)} \grecommentary[0pt]{Cf. Ap 19:1-2, 5-7 [AR]} \gresetinitiallines{1} \gresetlyriccentering{syllable}  \gregorioscore{chants/canticle--salus-et-honor--dom-1-et-3--english} \pagebreak %%this is here to help the rest of the spacing of the doc
 \subsection{Lectio brevis}     \hfill 2 Cor 1:3-4    \lettrine[lines=3]{P}{}raised be God, the Father of our Lord Jesus Christ, the Father of mercies and the God of all consolation! He comforts us in all our afflictions and thus enables us to comfort those who are in trouble, with the same consolation we have received from him.
 \subsection{Responsorium brevis}  \greannotation{VI} \index[Responsorium brevis]{Benedictus es, Domine} \label{Benedictus es, Domine (Responsorium brevis)} \grecommentary[0pt]{Dan 3:56 [AR]} \gresetinitiallines{1} \gresetlyriccentering{vowel}  \gregorioscore{chants/rb--benedictus_es_domine--solesmes}

 \subsection{Antiphona ad Magnificat}  \greannotation{II} \index[Antiphona ad Magnificat]{Vos estis lux mundi} \label{Vos estis lux mundi (Antiphona ad Magnificat)} \grecommentary[3pt]{Mt 5:14, 15 [AR]} \gresetinitiallines{1} \gresetlyriccentering{vowel} \grechangedim{maxbaroffsettextleft}{0 cm}{scalable} \gregorioscore{chants/an--vos_estis_lux_mundi--solesmes} \vspace{5pt} \emph{You are the light of the world. A city placed upon a mountain cannot be hidden; nor do they light a lamp and place it under a bushel, but upon a lampstand, that it might illuminate all who are in the house.} \grechangedim{maxbaroffsettextleft}{0.6 cm}{scalable}
   \subsection{Canticum Evangelicum} \subsubsection{The soul rejoices in the Lord} \greannotation{II}  \grecommentary[3pt]{Lc 1:46-55} \gresetinitiallines{1} \gresetlyriccentering{vowel} \gregorioscore{chants/magnificat2} \vspace{10pt} 

\emph{My soul proclaims the greatness of the Lord, my spirit rejoices in God my Savior for he has looked with favor on his lowly servant. From this day all generations will call me blessed: the Almighty has done great things for me, and holy is his Name. He has mercy on those who fear him in every generation. He has shown the strength of his arm, he has scattered the proud in their conceit. He has cast down the mighty from their thrones, and has lifted up the lowly. He has filled the hungry with good things, and the rich he has sent away empty. He has come to the help of his servant Israel for he has remembered his promise of mercy, the promise he made to our fathers, to Abraham and his children for ever. Glory to the Father, and to the Son, and to the Holy Spirit: as it was in the beginning, is now, and will be for ever. Amen.}


 \subsection{Preces}   \index[Preces]{Week I, Sunday, Second Vespers} \label{Week I, Sunday, Second Vespers (Preces)}    \lettrine[lines=2]{C}{}hrist the Lord is our head; we are his members. In joy let us call out to him: Lord, may your kingdom come.
\par \Rbar. Lord, may your kingdom come.

Christ our Savior, make your Church a more vivid symbol of the unity of all mankind, – make it more effectively the sacrament of salvation for all peoples. \par \Rbar. Lord, may your kingdom come.

Through your presence, guide the college of bishops in union with the Pope, – give them the gifts of unity, love and peace. \par \Rbar. Lord, may your kingdom come.

Bind all Christians more closely to yourself, their divine Head, – lead them to proclaim your kingdom by the witness of their lives. \par \Rbar. Lord, may your kingdom come.

Grant peace to the world, – let every land flourish in justice and security. \par \Rbar. Lord, may your kingdom come.

Grant to the dead the glory of resurrection, – and give us a share in their happiness. \par \Rbar. Lord, may your kingdom come.
  \subsection{Pater noster}   \index[Pater noster]{Pater noster} \label{Pater noster (Pater noster)}   \grechangedim{spaceabovelines}{0.4 cm}{scalable}  \gregorioscore{chants/or--pater_noster_a--solesmes-T} \grechangedim{spaceabovelines}{0 cm}{scalable}
  \subsection{Oratio conclusiva}   \index[Oratio conclusiva]{25th Sunday in OT} \label{25th Sunday in OT (Oratio conclusiva)}     \lettrine[lines=3]{K}{}eep your family safe, O Lord, with unfailing care, 
that, relying solely on the hope of heavenly grace,
they may be defended always by your protection.
Through our Lord Jesus Christ, your Son,
who lives and reigns with you in the unity of the Holy Spirit,
one God, for ever and ever. \par \Rbar. Amen.

 \subsection{Ritus conclusionis}        \par \Vbar. The Lord be with you. \par \Rbar. And with your spirit. \par \Vbar. May almighty God bless you, the Father, and the Son, and the Holy Spirit. \par \Rbar. Amen.
 \subsection{Benedicamus Domino}   \index[Benedicamus Domino]{Sundays} \label{Sundays (Benedicamus Domino)}      \gregorioscore{chants/misc.benedicamus.dominio.4-T}

\end{document}
